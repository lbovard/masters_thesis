\documentclass[letterpaper,12pt,titlepage,oneside,final]{book}
\newcommand{\package}[1]{\textbf{#1}} % package names in bold text
\newcommand{\cmmd}[1]{\textbackslash\texttt{#1}} % command name in tt font 
\newcommand{\href}[1]{#1} % does nothing, but defines the command so the
\usepackage{ifthen}
\newboolean{PrintVersion}
\setboolean{PrintVersion}{false} 
\usepackage{amsmath,amssymb,amstext,mathtools,amsfonts,amsthm} % Lots of math symbols and environments
\usepackage{bm}
\usepackage[pdftex]{graphicx} % For including graphics N.B. pdftex graphics driver 
\usepackage[pdftex,letterpaper=true,pagebackref=false]{hyperref} % with basic options
		% N.B. pagebackref=true provides links back from the References to the body text. This can cause trouble for printing.
\hypersetup{
    plainpages=false,       % needed if Roman numbers in frontpages
    pdfpagelabels=true,     % adds page number as label in Acrobat's page count
    bookmarks=true,         % show bookmarks bar?
    unicode=false,          % non-Latin characters in Acrobat’s bookmarks
    pdftoolbar=true,        % show Acrobat’s toolbar?
    pdfmenubar=true,        % show Acrobat’s menu?
    pdffitwindow=false,     % window fit to page when opened
    pdfstartview={FitH},    % fits the width of the page to the window
    pdftitle={uWaterloo\ LaTeX\ Thesis\ Template},    % title: CHANGE THIS TEXT!
%    pdfauthor={Author},    % author: CHANGE THIS TEXT! and uncomment this line
%    pdfsubject={Subject},  % subject: CHANGE THIS TEXT! and uncomment this line
%    pdfkeywords={keyword1} {key2} {key3}, % list of keywords, and uncomment this line if desired
    pdfnewwindow=true,      % links in new window
    colorlinks=true,        % false: boxed links; true: colored links
    linkcolor=blue,         % color of internal links
    citecolor=green,        % color of links to bibliography
    filecolor=magenta,      % color of file links
    urlcolor=cyan           % color of external links
}
\ifthenelse{\boolean{PrintVersion}}{   % for improved print quality, change some hyperref options
\hypersetup{	% override some previously defined hyperref options
%    colorlinks,%
    citecolor=black,%
    filecolor=black,%
    linkcolor=black,%
    urlcolor=black}
}{} % end of ifthenelse (no else)
\setlength{\marginparwidth}{0pt} % width of margin notes
\setlength{\marginparsep}{0pt} % width of space between body text and margin notes
\setlength{\evensidemargin}{0.125in} % Adds 1/8 in. to binding side of all 
\setlength{\oddsidemargin}{0.125in} % Adds 1/8 in. to the left of all pages
\setlength{\textwidth}{6.375in} % assuming US letter paper (8.5 in. x 11 in.) and 
\raggedbottom
\setlength{\parskip}{\medskipamount}
\renewcommand{\baselinestretch}{1} % this is the default line space setting
\let\origdoublepage\cleardoublepage
\newcommand{\clearemptydoublepage}{%
\clearpage{\pagestyle{empty}\origdoublepage}}
\let\cleardoublepage\clearemptydoublepage


\renewcommand{\Re}{\text{Re}}
\newcommand{\dt}{\Delta t}
\newcommand{\dx}{\Delta x}
\newcommand{\uhat}{\hat{\textbf{u}}}
\newcommand{\kvec}{\textbf{k}}
\newcommand{\uhatm}{$\hat{\textbf{u}}$ }
\newcommand{\kvecm}{$\textbf{k}$ }
%======================================================================
%   L O G I C A L    D O C U M E N T -- the content of your thesis
%======================================================================
\begin{document}

% For a large document, it is a good idea to divide your thesis
% into several files, each one containing one chapter.
% To illustrate this idea, the "front pages" (i.e., title page,
% declaration, borrowers' page, abstract, acknowledgements,
% dedication, table of contents, list of tables, list of figures,
% nomenclature) are contained within the file "uw-ethesis-frontpgs.tex" which is
% included into the document by the following statement.
%----------------------------------------------------------------------
% FRONT MATERIAL
%----------------------------------------------------------------------
%% T I T L E   P A G E
% -------------------
% Last updated May 24, 2011, by Stephen Carr, IST-Client Services
% The title page is counted as page `i' but we need to suppress the
% page number.  We also don't want any headers or footers.
\pagestyle{empty}
\pagenumbering{roman}

% The contents of the title page are specified in the "titlepage"
% environment.
\begin{titlepage}
        \begin{center}
        \vspace*{1.0cm}

        \Huge
        {\bf Short-Wave Vortex Instabilities in Stratified Flow}

        \vspace*{1.0cm}

        \normalsize
        by \\

        \vspace*{1.0cm}

        \Large
        Luke Bovard \\

        \vspace*{3.0cm}

        \normalsize
        A thesis \\
        presented to the University of Waterloo \\ 
        in fulfillment of the \\
        thesis requirement for the degree of \\
        Master of Mathematics\\
        in \\
        Applied Mathematics \\

        \vspace*{2.0cm}

        Waterloo, Ontario, Canada, 2013 \\

        \vspace*{1.0cm}

        \copyright\ Luke Bovard 2013 \\
        \end{center}
\end{titlepage}

% The rest of the front pages should contain no headers and be numbered using Roman numerals starting with `ii'
\pagestyle{plain}
\setcounter{page}{2}

\cleardoublepage % Ends the current page and causes all figures and tables that have so far appeared in the input to be printed.
% In a two-sided printing style, it also makes the next page a right-hand (odd-numbered) page, producing a blank page if necessary.
 


% D E C L A R A T I O N   P A G E
% -------------------------------
  % The following is the sample Delaration Page as provided by the GSO
  % December 13th, 2006.  It is designed for an electronic thesis.
  \noindent
I hereby declare that I am the sole author of this thesis. This is a true copy of the thesis, including any required final revisions, as accepted by my examiners.

  \bigskip
  
  \noindent
I understand that my thesis may be made electronically available to the public.

\cleardoublepage
%\newpage

% A B S T R A C T
% ---------------

\begin{center}\textbf{Abstract}\end{center}
Density stratification \francis{is one of the essential underlying physical mechanisms} for atmospheric and oceanic flow. As a first step to investigating the mechanisms of stratified turbulence, linear stability plays a critical role in determining under what conditions a flow remains stable or unstable. In the study of transition to stratified turbulence, a common vortex model, known as the Lamb-Chaplygin dipole, is used to investigate the conditions under which stratified flow transitions to turbulence. Numerous investigations have determined that a critical length scale, known as the buoyancy length, plays a key role in the breakdown and transition to stratified turbulence. At this buoyancy length scale, an instability unique to stratified flow, the zigzag instability, emerges. However investigations into sub-buoyancy length scales have remained unexplored. In this thesis we discover and investigate a new instability of the Lamb-Chaplyin dipole that exists at the sub-buoyancy scale. \francis{By numerically performing the linear stability analysis} we show that this short-wave instability exhibits growth rates similar to that of the zigzag instability. We conclude with nonlinear studies of this short-wave instability and demonstrate this new instability saturates at a level proportional to the cube of the aspect ratio. 
\cleardoublepage
%\newpage

% A C K N O W L E D G E M E N T S
% -------------------------------

\begin{center}\textbf{Acknowledgements}\end{center}
Thanks to my supervisor Dr. Michael Waite. Additional thanks goes to Dr. Marek Stastna and Dr. Francis Poulin. 

Financial support for this work was provided by the Natural Sciences and Engineering Research Council of Canada. Computations were made possible by the facilities of the Shared Hierarchical Academic Research Computing Network (SHARCNET) and Compute/Calcul Canada.

Obligatory thanks to everyone else who I have learnt from and talked to over all these year. There are too many to name, so I won't. Just know that if you knew me I'd probably thank you. This saves me having to remember people and just cover everyone in one full swoop.

The work for this thesis was written under the music of, in order of most played, Queen, Taylor Swift, The Beatles, Dire Straits, and Billy Joel. I kept statistics and it works out to be a few thousands plays of each, which is probably about 4 days straight of music per artists!

Richard you wanted me to add you punching Chris by mistake many years ago to this thesis. Here you go.

Special thanks to Patricia for conversations during the last few months. It made writing this thesis much easier. Thanks et merci y gracias por todo. 

Finally, and most importantly, I dedicate this thesis to the two best teachers I've ever had. Without both of them, who knows where I'd be now. 
\cleardoublepage
%\newpage

% D E D I C A T I O N
% -------------------

\begin{center}\textbf{Dedication}\end{center}
\begin{center}To Dr. Edward Vrscay and Dr. Jamal Sakhr.\end{center}
\cleardoublepage

%\newpage

% T A B L E   O F   C O N T E N T S
% ---------------------------------
\renewcommand\contentsname{Table of Contents}
\tableofcontents
\cleardoublepage
\phantomsection
%\newpage

% L I S T   O F   T A B L E S
% ---------------------------
%\addcontentsline{toc}{chapter}{List of Tables}
%\listoftables
%\cleardoublepage
%\phantomsection		% allows hyperref to link to the correct page
%%\newpage

% L I S T   O F   F I G U R E S
% -----------------------------
\addcontentsline{toc}{chapter}{List of Figures}
\listoffigures
\cleardoublepage
\phantomsection		% allows hyperref to link to the correct page
%\newpage

% L I S T   O F   S Y M B O L S
% -----------------------------
% To include a Nomenclature section
% \addcontentsline{toc}{chapter}{\textbf{Nomenclature}}
% \renewcommand{\nomname}{Nomenclature}
% \printglossary
% \cleardoublepage
% \phantomsection % allows hyperref to link to the correct page
% \newpage

% Change page numbering back to Arabic numerals
\pagenumbering{arabic}



%----------------------------------------------------------------------
% MAIN BODY
%----------------------------------------------------------------------
% Because this is a short document, and to reduce the number of files
% needed for this template, the chapters are not separate
% documents as suggested above, but you get the idea. If they were
% separate documents, they would each start with the \chapter command, i.e, 
% do not contain \documentclass or \begin{document} and \end{document} commands.
%======================================================================
%\chapter{Introduction}
%\begin{itemize}
%	\item Brief introduction to the topic
%\end{itemize}
%%======================================================================
%THIS IS THE BEGINNING


%\chapter{Fluids Background}
\section{Boussinesq approximation}


The equations can be written in the dimensional from as (add derivation, Kundu? Batchelor? Others?)

\begin{align}
\frac{\partial \bm{u}}{\partial t} + \bm{u}\cdot \nabla \bm{u} = -\frac{1}{\rho_{0}}\nabla p - \frac{\rho g}{\rho_{0}}\hat{\bm{e}}_{z} + \nu \nabla^{2}\bm{u} \label{boussinesq1}\\
\nabla \cdot \bm{u} =0 \label{boussinesq2}\\
\frac{\partial \rho}{\partial t} + \bm{u}\cdot \nabla \rho = \kappa \nabla^{2}\rho - \frac{\partial \bar{\rho}}{\partial z} w\label{boussinesq3}
\end{align}
where we have the following dimensional variables
\begin{itemize}
\item $\textbf{u}(x,y,z)=(u(x,y,z),v(x,y,z),w(x,y,z))$ is the velocity field in the $x,y,z$-directions directions respectively,
\item $p(x,y,z)$ is the pressure field,
\item $\rho(x,y,z)$ is a perturbation density,
\item $\rho_{0}$ is a constant reference density,
\item $\bar{\rho}$ is ... ? ,
\item $g$ is the gravitational constant,
\item $\nu$ is the constant kinematic velocity ,
\item $\kappa$ is the constant molecular diffusivity.
\end{itemize}
Equations (\ref{boussinesq1})-(\ref{boussinesq3}) are a set of five coupled partial differential equations for the unknowns $\textbf{u},p,\rho$. Herein, we will refer to equation (\ref{boussinesq1}) as the velocity equations, equation (\ref{boussinesq2}) as the continuity equation, and equation (\ref{boussinesq3}) as the density equation. 

Maybe go onto non-dimensionalisation? 

There is a difference between the linear and nonlinear equations (MIke's code uses a slightly modified version of Boussinesq that uses temperature) 
We take the above equations (Ref) as the starting point for our simulations. 

%-------------------------------------------------------------------------------
%
%  NUMERICAL STUFF AND BOUSSINESSQ STUFF
%
%-------------------------------------------------------------------------------
\chapter{Numerical Background}
%\begin{itemize}
%\item introduction to the boussinesq approximation 
%\item In this section I should introduce the spectral method and the numerical scheme
%\item Dealiasing tests 
%\item  diffusion test
%\item hyperviscosity
%\end{itemize}

In this chapter we discuss the numerical techniques and methods used in this thesis. In this thesis we use the spectral method to numerically solve the linear and nonlinear Navier-Stokes equations. Spectral methods provide a convenient and quick way to compute derivatives of sufficiently well-behaved periodic functions. In evaluating derivatives of smooth periodic functions, spectral methods provide an advantage over other methods of evaluating derivatives, such as finite difference, as the derivatives can be computed to with greater accuracy, often to machine precision, in exchange for only a relatively modest increase in complexity. Specifically, finite difference methods run in $\mathcal{O}(N)$ time, where $N$ is the number of grid points, but the error tends to be on the order of $\mathcal{O}(N^{-p})$ where $p$ is a small integer \cite{durran}. Spectral methods, however, run in $\mathcal{O}(N\log N)$ time but the errors are on the order of $\mathcal{O}(N^{-p})$ when the function is $C^{p}$\cite{trefethen_spectral}. A complete overview of spectral methods is beyond the scope of this thesis and we only discuss the key features needed for numerical work. Comprehensive reviews of spectral methods are provided in many books, e.g. Trefethen \cite{trefethen_spectral}, Boyd \cite{boyd2001}, and Canuto \cite{canuto}.


\section{Spectral Methods Motivation}
Spectral methods have their origins in the Fourier transform. Let us denote the Fourier transform, $\mathcal{F}$, of $f(x)$ as $\mathcal{F}(f(x)) = \hat{f}(k)$, given by
\begin{align}
\mathcal{F}(f(x)) = \hat{f}(k) = \int_{-\infty}^{\infty}dxe^{-ikx}f(x)\label{fouriertransform}.
\end{align}
Now consider the Fourier transform of the derivative $df/dx$: 
\begin{align} 
\mathcal{F}\left(\frac{df}{dx}\right)= \int_{-\infty}^{\infty}dxe^{-ikx}\frac{df}{dx}=e^{-ikx}f(x)\bigg|_{-\infty}^{\infty} + ik\int_{-\infty}^{\infty}dxe^{-ikx}f(x)= ik\hat{f}(k),
\end{align}
and thus the Fourier transform of a derivative is just $ik$ times the Fourier transform of $f(x)$. An important assumption here that $f(x)$ vanishes sufficiently quickly at infinity otherwise the $e^{-ikx}f(x)$ term is non-negligible. For most applications of the Fourier transform, but not all, this assumption is valid. For a complete treatment of the conditions of the existence of the Fourier transform, see e.g. Kammler \cite{kammler}. With this result in hand, it is easy to show via induction that the Fourier transform of $d^{n}f/dx^{n}$ is $(ik)^{n}\hat{f}(k)$. Hence, once we have the Fourier transform $\hat{f}(k)$, the $n$-th derivative is obtained by computing the inverse Fourier transform
\begin{align}
\frac{d^{n}f}{dx^{n}} = \frac{1}{2\pi} \int_{-\infty}^{\infty}dx e^{ikx}(ik)^{n}\hat{f}(k).\label{inversefouriertransform}
\end{align}

This procedure suggests a method to compute the derivative of a function. Computationally, if we have a way to evaluate $\hat{f}(k)$ from $f(x)$ and $f(x)$ from $\hat{f}(k)$, then the $n$-th derivative is easy to compute via the following algorithm:  
\begin{enumerate} 
\item Compute $\hat{f}(k)$ from $f(x)$
\item Multiply $\hat{f}(k)$ by $(ik)^{n}$ 
\item Invert $(ik)^{n}\hat{f}(k)$ to obtain $f^{(n)}(x)$
\end{enumerate}

We need a method to evaluate the integrals (\ref{fouriertransform}) and (\ref{inversefouriertransform}). One possible avenue would be to apply well known integral quadratures to evaluate these integrals. However, there are more efficient and better alternatives. The correct avenue to implementing the above algorithm is to consider the discrete analogue of the Fourier transform, the discrete Fourier transform. From the discrete Fourier transform, we discuss the Fast Fourier Transform, which is a fast way to evaluate the discrete Fourier transform.
\section{FFT and Spectral Derivatives} 
Let us now investigate how we can quickly compute the Fourier transform of a function on a computer. In analogy to the Fourier transform, we define \cite{trefethen_spectral} the discrete Fourier transform\footnote{Note that there is no universal standard on where to put the factors of $N$ and $2\pi$.} (DFT) on $N$ data points $v_{j}$
\begin{align}
\hat{v}_{k} = \frac{2\pi}{N}\sum_{j=1}^{N} e^{-ikx_{j}}v_{j},\qquad k=-\frac{N}{2}+1,\ldots,\frac{N}{2},\label{dft}
\end{align}
and the inverse discrete Fourier transform (IDFT)  on $N$ data points $\hat{v}_{k}$
\begin{align}
v_{j} = \frac{1}{2\pi}\sum_{k=-N/2+1}^{N/2} e^{ikx_{j}}\hat{v}_{k}, \qquad j=1,\ldots, N.\label{idft}
\end{align}
where $x_{j}\in[0,2\pi]$. Here we take $N$ to be even to simplify the formulas, however all results hold for odd $N$ with slight modification. The range of the wavenumbers is from $-N/2+1$, not $-N/2$, due to the $2\pi$ periodicity of $v_{j}$. Wavenumbers $-N/2$ are equivalent to $N/2$ and we do not want to count this wavenumber twice. 


From the definitions of the DFT and inverse DFT, we can see a close analogy to the continuous Fourier and inverse Fourier transforms. It can be shown \cite{trefethen_spectral} that these are the correct discrete analogies. 

In order to approximate the derivative of a function, we first need to determine how to sample the function $f(x)$. We assume $f(x)$ to periodic on $0\le x\le 2\pi$. If the domain is different, a simple scaling can transform the periodic domain to this standard interval. To compute the derivative of $f(x)$, we first sample at the points 
\begin{align}
x_{j} = \frac{2\pi j}{N}, j=1,\ldots,N
\end{align}
and set $v_{j}=f(x_{j})$. Now we can compute the DCT from $v_{j}$ giving the coefficients $\hat{v}_{k}$ which is the discrete analogue of $\hat{f}(k)$.  Since this is the analogue of the Fourier transform, we now multiply by $ik$ resulting in the analogue of computing the first derivative. Finally, we compute the inverse DFT from $ik\hat{v}_{k}$ which produces the spectral approximation, $v'_{j}$, to the derivative of the sampled function $f'(x)$ at points $x_{j}$. However, there is some subtlety involved in treating the wavenumber $N/2$, which is usally set to zero \cite{trefethen_spectral}.

Despite having a discrete algorithm to compute an analogue of the Fourier transform, our goal has still not been reached because we are still left with doing $\mathcal{O}(N^{2})$ operations. This results because there are $N$ coefficients and for each coefficient a sum of $N$ terms must be added together. For small $N$, computing spectral derivatives are not that much slower than finite differences methods, however they quickly become impractical for large $N$. This barrier can be overcome through the Fast Fourier Transform (FFT) algorithm. This algorithm has a rich history and has been rediscovered multiple times, beginning with Gauss in the early 1800s before Fourier published his results on Fourier series \cite{kammler}. The most popular and well known rediscovery of the FFT was in 1965 by Cooley-Tukey \cite{cooley1965}.

The FFT employs a divide and conquer algorithm. The basic idea is to notice that computing the discrete Fourier transform can be divided into even and odd terms which can be computed independently of each other. Thus assuming $N$ is a power of two, we have two new discrete Fourier transform problems of size $N/2$. Each other those problems can themselves be decomposed into problems of size $N/4$. Repeating this process recursively we are able to divide the computation of the DFT into $\mathcal{O}(\log N)$ problems. Computing the DFT of each problem takes roughly $\mathcal{O}(N)$ and thus the total running time is on the order of $\mathcal{O}(N\log N)$. In this simple case, it is easy to prove rigorously from the recursion relationship that the running time is $\mathcal{O}(N\log N)$ using the Master theorem \cite{clrs}.

$\mathcal{O}(N\log N)$ is a general result that holds for all values of $N$, but the proofs are more complicated and in most applications, $N$ is chosen to be a power of small primes for which their exist many well optimised and developed algorithms. The derivation and implementation of the FFT for general values of $N$ is an interesting and complicated question that has sparked a lot of research into the best way to decompose the problem. Additionally, the actual implementation details can differ depending on the value of $N$, the type of computer, and the type of data being considered. It is beyond the scope of the thesis to discuss such implementation details and further technical and hardware details that can be used in implementations can be found in \cite{fftw}.

Now that we have an algorithm to compute the FFT, the inverse Fast Fourier Transform (IFFT) can easily be derived since the form of (\ref{idft}) is very similar to that of (\ref{dft}) and the same strategy of divide and conquer can be applied. With this we have the following algorithm for computing the spectral derivative:
\begin{enumerate} 
\item Sample function $f(x)$ at $N$ data points $x_{j}$ to obtain $v_{j}, j=1,\ldots,N$.
\item Compute, using the FFT, $\hat{v}_{k}$ from $v_{j}$.
\item Multiply $\hat{v}_{k}$ by $(ik)^{n}$ to compute the $n$-th derivative.
\item Invert $(ik)^{n}\hat{v}_{k}$ using the IFFT  to obtain an approximation to $f^{(n)}(x)$ at $x_{j}$. 
\end{enumerate}

To illustrate spectral derivatives, we follow \cite{trefethen_spectral} and demonstrate spectral differentiation using two simple examples. Consider the following two periodic functions
\begin{align}
f(x) = \max(0,1-|x-\pi|/2), \qquad g(x)=e^{\sin x}
\end{align}
which are both periodic on the interval $[0,2\pi]$. 
\begin{figure}
\begin{center}
\includegraphics[width=\textwidth]{spectral_derivatives.pdf}
\caption{Spectral differentiation of two functions $\max(0,1-|x-\pi|/2)$ (top) and $e^{\sin x}$ (bottom) from \cite{trefethen_spectral}. On the right is the spectral derivative computed with $n=24$. Red curve represents the exact derivative. Even as such small $N$, the derivative of $e^{\sin x}$ is very smooth and the error is on the order of $\mathcal{O}(10^{-13})$. The spectral derivative of $\max(0,1-|x-\pi|/2)$ is not as good due to the discontinuity at $x=\pi$.}
\label{spectral_derivatives}
\end{center}
\end{figure}

As we can see in Fig.~\ref{spectral_derivatives} when the original function is sufficiently smooth and periodic, here $g(x)$, the accuracy of the derivative is very good. However, when the function is not smooth, as exhibited in $f(x)$, the accuracy is not very good. Here $g(x)$ is not differentiable at $\pi$ and spectral methods have a difficult time dealing with this. For a complete discussion of the specific smoothness and accuracy is contained in Chapter 4 of \cite{trefethen_spectral}.

In this section, we have focused only on simple one-dimensional examples, but the method easily generalises to arbitrary dimensions. To compute multiple partial derivatives, the only change to the algorithm is that we have seperate wavenumbers for the different directions. The only modification is computing the higher dimensional FFTs. To compute the $n$-dimensional FFT of a given function, we must compute the FFT of the all the columns and all the rows in each dimension\cite{kammler}. For example in 2D dimensions, we would compute the FFT of each of the columns, and then computing the FFT of all the rows of the previous result \cite{matlabfft2}. 

\section{Dealiasing} 
In the previous section we demonstrated that spectral differentiation can produce very accurate results in only $\mathcal{O}(N\log N)$ operations. However, we have only considered the spectral derivative of a single function $f(x)$. The question naturally arises about what happens if we have products of functions, for example advection like terms in the Navier-Stokes equations 
\begin{align}
u\frac{\partial u}{\partial x} + v\frac{\partial u}{\partial y} +w\frac{\partial u}{\partial z}
\end{align}
where we now have a product of a function and a derivative. Such terms are known as a convolution and require a more careful treatment. In this section, we introduce two closely related methods for evaluating expressions of the above form, the spectral method and the pseudospectral method. For this we carefully follow the treatment of Durran \cite{durran}. 

Consider the following general 1D non-linear PDE
\begin{align}
\frac{\partial \psi}{\partial t} + F(\psi) = 0
\end{align}
where $F(\psi)$ is some general nonlinear function. Consider seeking a series expansion of the form \cite{durran}
\begin{align}
\psi(x,t)\approx \phi(x,t) = \sum_{k=1}^{N}a_{k}(t)\varphi_{k}(x)
\end{align}
where $\varphi_{k}$ are basis functions around which we are interested in expanding the solution. Some examples of such functions might be complex exponentials, Bessel functions, or spherical harmonics. In general,  we will not be able to find the eigenfunctions that provide the natural basis to seek a series expansion. Without the proper basis functions, it is clear that the approximate solution will never exactly solve the original PDE and we have to determine a way to minimise the error. For many problems, there is often some degree of symmetry so choosing a certain basis makes sense. For example, for a problem within a periodic box, it makes sense to choose complex exponentials as a basis; if the problem is based on a sphere, spherical harmonics are a natural choice. Thus, we are interested in not finding the correct $\varphi_{k}$ for a given problem but instead focus picking a known basis and appropriately choosing $a_{k}(t)$ to minimise the residual \cite{durran},
\begin{align}
R(\phi) = \frac{\partial \phi}{\partial t} + F(\phi),
\end{align}
in some way.  

\subsection{Spectral Method}
How we choose to minimise this error leads to different methods of solving the problem. Right now we consider the so-called ``spectral method" which is a special case of a very general technique called Galerkin approximation.  The spectral method requires the residual be orthogonal to the basis functions $\varphi_{k}$, i.e.
\begin{align}
\int_{S}R(\phi(x))\varphi_{k}(x)dx = 0 \qquad k=1,\ldots,N.\label{residual_constraint}
\end{align}
For the special case of the spectral method, we require that the basis functions be orthogonal. Other basis functions that are not orthogonal but satisfy other constraints, such as support over very small areas, lead to a different approximation known as the finite element method\cite{durran}. It can be shown \cite{durran} that (\ref{residual_constraint}) is equivalent to minimising the $L^{2}$ error of the residual. 

By imposing the above integral, it can be shown\cite{durran} that the resulting system of ODEs for the above is
\begin{align}
\frac{d a_{k}}{dt} = -\frac{1}{M_{k,k}}\int_{S}\left[F\left(\sum_{n=1}^{N}a_{n}\varphi_{n}\right)\varphi_{k}\right]dx \qquad k=1,\ldots,N
\end{align}
where $M_{k,k}=\langle \varphi_{k},\varphi_{k}\rangle$. For our work, we are interested in a specific case of basis functions, $\varphi_{k}(x)=e^{ikx}$ and the inner product to be the standard inner product for complex functions
\begin{align}
\int_{S}g(x)h^{*}(x)dx=0 
\end{align}

To demonstrate the above formulation, we use the spectral method to solve the 1D advection equation with variable windspeed,
\begin{align}
\frac{\partial\phi}{\partial t} + c(x,t)\frac{\partial \phi}{\partial x} =0, F(\phi) = c(x,t)\frac{\partial \phi}{\partial x}.
\end{align}
We now expand out $\phi(x,t)$ as the sum of $N=2K+1$ Fourier modes, where here we are now letting $N$ be an odd number in contrast to the previous section where $N$ was even. This is chosen because it simplifies the formulas, however the results hold for even $N$ as well, but with some slight alterations to the algebra. Expanding
\begin{align}
\phi(x_{j},t)= \sum_{n=K}^{K}a_{n}(t)e^{inx_{j}},
\end{align}
and plugging this into the above yields
\begin{align}
\frac{d a_{k}}{dt} = -\frac{i}{2\pi}\sum_{n=-K}^{K}na_{n}\int_{-\pi}^{\pi}c(x,t)e^{i(n-k)x}dx \qquad k=-K,\ldots,K.
\end{align}
Expanding out $c(x,t)$ as a Fourier series we obtain
\begin{align} 
\frac{d a_{k}}{dt} = -\frac{i}{2\pi}\sum_{n=-K}^{K}\sum_{m=-K}^{K}na_{n}c_{m}\int_{-\pi}^{\pi}e^{i(n+m-k)x}dx \qquad k=-K,\ldots,K
\end{align}
and upon using the orthogonality of the integral, we obtain
\begin{align}
\frac{da_{k}}{dt} = -\sum_{\substack{m+n=k\\ |m|,|n|\le K}} inc_{m}a_{n}
\end{align}
where we require that $n+m=k$ and $|n|,|m|\le K$. If we want to evaluate this sum directly, we would need to evaluate $\mathcal{O}(K^{2})$ operations due to the double sum in the convolution. Historically, this was a barrier for spectral methods because even though they provided accurate results, this $\mathcal{O}(K^{2})$ bottleneck did not allow for larger numerical simulations, where $\mathcal{O}(K)$ finite differences method could \cite{durran}.

In order to get around this bottleneck, we can turn the $\mathcal{O}(K^{2})$ to $\mathcal{O}(K\log K)$ using the FFT. The general idea is as follows: transform the Fourier coefficients to real space in $\mathcal{O}(K\log K)$, multiply the two terms together at each grid point in $\mathcal{O}(K)$, and transform back to Fourier space in $\mathcal{O}(K\log K)$. The resulting algorithm is $\mathcal{O}(K\log K)$. However, if we were to implement this naively, errors would arise, due to the phenomena of aliasing. 

At a basic level, aliasing is the result of multiplying two terms grid-pointwise that generate short waves that can be aliased into long waves. This arises in the spectral method when we compute the product of two Fourier coefficients. A very similar, but exaggerated example, is known as the wagon wheel effect and can illustrate the general idea of short waves being interpreted as long waves \cite{wagonwheel}.  The wagon wheel effect arises when an object appears to be stationary but is really moving. For example, imagine a camera that takes a picture every second of a turbine, but the turbine is moving at 10 revolutions per second. For someone looking at the shots from the camera, they would claim the turbine isn't moving. If the turbine is moving slightly faster, say 10.1 revolutions per second, the camera would show that the turbine is moving, albeit rather slowly. If it were moving slightly slower, say 9.9 revolutions per second, the camera would show the turbine moving backwards. Thus one would interpret the turbine as moving very slowly in either case and thus conclude that it must have a very slow period, or long wavelength. In actuality, the period is much more rapid and the wavelength of one revolution is much shorter.  
\begin{figure}
\begin{center}
\includegraphics[width=\textwidth]{aliassine.pdf}
\caption{Different sine curves that fit the same set of data points \cite{sinewiki}. This is an example of aliasing.}
\label{aliassine}
\end{center}
\end{figure}

Mathematically, we can see a similar phenomena by sampling a sine wave. Consider Figure~\ref{aliassine} which has 11 points. The short wavelength red sine curve samples these points with a short wavelength while a longer wavelength blue curve also samples, despite being non-existent in the actual data. The goal is to minimise this effect. 

To illustrate this consider two functions $f(x),g(x)$. We want to compute the Fourier coefficients of the product
\begin{align}
f(x)g(x) = \sum_{m=-K}^{K} p_{k}e^{ikx},
\end{align}
and we can expand $f(x),g(x)$ as Fourier series
\begin{align}
f(x) = \sum_{m=-K}^{K}a_{m}e^{imx}, \qquad g(x) = \sum_{n=-K}^{K}b_{n}e^{inx}.
\end{align}
The above algorithm states that we convert the Fourier coefficients $a_{m},b_{n}$ to real space and multiply. Suppose that in real space the spacing between the grid points is
\begin{align}
x_{j} = \frac{2\pi j}{2N+1}
\end{align}
where we let $2N+1$ denotes the total number of grid points. 

How do we choose $N$? Naively, it seems logical to set that $N=K$ so that both real and Fourier space have the same number of co-efficients. This naive choice leads to aliasing error and instead we must choose $N>K$ to avoid this aliasing error. To get intuition for why, aliasing error arises when two poorly resolved waves are multiplied together and produce a longer wavelength wave, as discussed above in Figure~\ref{aliassine}. We can avoid this if all short wavelengths are resolved properly.

To find the optimal value of $N$, consider multiplying two functions given by Fourier modes, each with the largest wavenumber $K$. This product generates a wave with wavenumber $2K$, recalling that multiplication of waves is addition of the wavenumbers. The largest possible resolvable wave in the real space grid is $\pi/\dx=N+1/2$. Here we can explicitly see what happens if $K=N$. The $2K=2N$ waves would not be resolved properly because the largest possible wavenumber to be resolved is only $N$. An obvious choice to fix this is to choose $N=2K$ and thus we could resolve all possible wavenumbers. However this choice is too loose a bound since it would imply that for every simulation, we would require twice as many grid space coefficients as Fourier coefficients. It is possible to do better. 

To do better, consider which wavenumber the $2K$ would be aliased into, which will be $2K-2\pi/\dx$ thus
\begin{align}
K<|2K- \frac{2\pi}{\dx}|,
\end{align}
and upon subbing in $\dx$ we find that 
\begin{align}
N > \frac{3}{2}K - 1
\end{align}

There is an alternative way to get at the same result. Consider the discrete Fourier transform of the product $f(x)g(x)$
\begin{align}
p_{k} = \frac{1}{2N+1}\sum_{j=1}^{2N+1}f(x_{j})g(x_{j})e^{-ikx_{j}},
\end{align}
and now plug in the 
\begin{align}
f(x) = \sum_{m=-N}^{N}a_{m}e^{imx}, \qquad g(x) = \sum_{n=-N}^{N}b_{n}e^{inx},
\end{align}
where we have replaced $K$ with $N$ where $N>K$. This is the same expansion as before but we have set $a_{l},b_{l}=0$ if $K < |l|\le N$. Plugging these into the above yields
\begin{align}
p_{k} = \frac{1}{2N+1}\sum_{m=-N}^{N}\sum_{n=-N}^{N}\sum_{j=1}^{2N+1}a_{m}b_{n}e^{i(m+n-k)x_{j}}.
\end{align}
Recalling that $x_{j} = 2\pi j/(2N+1)$ the inner summation is
\begin{align}
\sum_{j=1}^{2N+1}e^{i(m+n-k)x_{j}} =  \begin{cases}
1 & m+n-k =0\\
1 & m+n-k =2N+1\\
1 & m+n-k =-(2N+1)\\
0 & \text{otherwise}
\end{cases}
\end{align}
which is a restatement of the well known orthogonality condition \cite{durran}. Thus we can break up the above summation into three cases, $m+n=k, m+n=k\pm(2N+1)$ which gives
\begin{align}
p_{k} = \sum_{\substack{m+n=k\\ |m|,|n|\le N}}a_{m}b_{n}+  \sum_{\substack{m+n=k+2N+1\\ |m|,|n|\le N}}a_{m}b_{n}+ \sum_{\substack{m+n=k-2N-1\\ |m|,|n|\le N}}a_{m}b_{n}.
\end{align}
Note the similarity with the result from the spectral method, which is identical to summation encountered in the spectral method, recalling that the coefficients are zero for values greater then $K$. The aliasing error arises from the second two terms. We want to eliminate this error so we have to decide how to choose $N$ to eliminate this term. We consider the first aliasing error term. We have that if $m,n>K$ then $a_{m}b_{n}=0$. It follows that $m+n>2K$ this term will vanish, i.e. $m+n=k+2N+1>2K$ for all $k=-K,\ldots,K$. Thus we want to consider $k=-K$ since this is the smallest possible value of this summation, i.e. $-K+2N+1>2K$. Rearranging this yields $N>(3K-1)/2$ as before. An identical argument gives the same result for the other aliasing term. 

This dealiasing rule is known as the ``two-thirds rule" and will completely eliminate aliasing effects. The above algorithm tells us how to choose $N$ given $K$, however in many applications we typically work the other way around, choosing $N$ and limiting $K$. We now can now write down the algorithm for evaluating the following term $c(x)\partial\phi/\partial x$ in Fourier space, when we are initially only given $M$ samples: 
\begin{itemize}
\item Sample $c(x),\phi(x,t)$ at $M$ points.
\item Compute the Fourier coefficients of $c(x),\phi(x,t)$ using the FFT yielding $M$ Fourier coefficients $a_{i},b_{j}$ respectively.
\item Evaluate the spectral derivative of $\phi$ by multiplying the coefficients by $ik$, i.e $a_{l}=ika_{l}$.
\item Modify the Fourier coefficients so the top $1/3$ are zero. 
\item Take the IFFT back to real space.
\item Multiply the real space coefficients gridwise.
\item Take the FFT of this product.
\end{itemize}
This procedure will allow us to evaluate the Navier-Stokes equations in Fourier space by evaluating convolution terms by multiplying fields in real space and to eliminate numerical error. In the next section, we provide a simple example that illustrates this method in practice.

\subsection{Pseudospectral Method}
We have so far been focusing on eliminating the aliasing error at the cost of eliminating some Fourier modes which can be achieved by taking $N>3K/2$. A closely related method known as the pseudospectral method is an alternative way to evaluate the above equations. In this case, instead of enforcing that the residual is orthogonal to the basis functions, we instead choose a collocation method \cite{durran}, i.e.
\begin{align}
R(\phi(j\dx)) = 0,
\end{align}
where we enforce that the coefficients satisfying the equation at each grid point. Let us now expand $\phi(x,t),c(x,t)$ are Fourier series
\begin{align}
\phi(x,t) = \sum_{n=-K}^{K}a_{n}e^{inx}, \qquad c(x,t) = \sum_{m=-K}^{K}c_{m}e^{imx}.
\end{align}
Substituting into the test equation and enforcing the collocation requirement yields
\begin{align}
\sum_{n=-K}^{K}\frac{da_{n}}{dt}e^{inx_{j}} + \sum_{m=-K}^{K}c_{m}e^{imx_{j}}\sum_{n=-K}^{K}ina_{n}e^{inx_{j}}=0.
\end{align} 
which we can alternatively write as
\begin{align}
\frac{d\phi_{j}}{dt} + c_{j}\frac{\partial \phi_{j}}{\partial x} = 0,
\end{align}
where we have written
\begin{align}
\phi(x_{j}) = \sum_{n=-K}^{K}a_{n}e^{inx_{j}}, c(x_{j}) = \sum_{n=-K}^{K}c_{n}e^{inx_{j}}, \frac{\partial\phi_{j}}{\partial x} = \sum_{n=-K}^{K}ina_{n}e^{inx_{j}}.
\end{align}
This is the set of equations we want to solve. Thus we now have a way to evaluate using the following algorithm:
\begin{enumerate}
\item Evaluate $\phi(x,t),c(x,t)$ at $M$ points.
\item Compute the spectral derivative of $\phi(x,t)$ using the methods of the previous sections.
\item Multiply all the coefficients gridpoint wise.
\item Timestep the resulting solution.
\end{enumerate} 

\subsection{Spectral vs Psuedospectral}
The equations resulting from the pseudospectral method are actually very similar to the spectral equations except that there is no dealiasing. To see why, multiply the above equation by $e^{-ikx_{j}}$ and sum over all $j$ and we will obtain
\begin{align}
\frac{da_{k}}{dt}= -\sum_{\substack{m+n=k\\ |m|,|n|\le K}}inc_{m}a_{n}-\sum_{\substack{m+n=k+2K+1\\ |m|,|n|\le K}}inc_{m}a_{n}- \sum_{\substack{m+n=k-2K-1\\ |m|,|n|\le K}}inc_{m}a_{n}.
\end{align}
For comparison, the spectral method equations are
\begin{align} 
\frac{da_{k}}{dt} = -\sum_{\substack{m+n=k\\ |m|,|n|\le K}} inc_{m}a_{n},
\end{align}
where in both equations $k=-K.\ldots,K$. 

There are many similarities between these two equations but we have to remember that they were derived in two completely different ways. Recall that to evaluate the convolution term in the spectral method, we obtained the following equation
\begin{align}
p_{k} = \sum_{\substack{m+n=k\\ |m|,|n|\le N}}a_{m}b_{n}+  \sum_{\substack{m+n=k+2N+1\\ |m|,|n|\le N}}a_{m}b_{n}+ \sum_{\substack{m+n=k-2N-1\\ |m|,|n|\le N}}a_{m}b_{n}.
\end{align}
which has the exact same form as the pseudospectral method but the crucial difference is that the sum is over $N$ and not $K$. If we set $N=K$, which would result in dealiasing error, we would obtain the pseudospectral method. Thus, at a fundamental level, the pseudospectral method is just the spectral method without dealiasing.

\section{Timestepping and Examples}
Recall that our model 1D problem is 
\begin{align}
\frac{\partial \psi}{\partial t} + F(\psi) = 0.
\end{align}
We have been focusing on evaluating the $F(\psi)$ term but now we turn to the time derivative term. Fortunately, this term is much simpler to deal with than the nonlinear terms since it appears only as a single derivative. 

To evaluate this term, consider the approximation to the first derivative \cite{durran}
\begin{align}
\frac{\partial \psi}{\partial t} \approx  \frac{\psi^{n+1}-\psi^{n}}{\Delta t}.
\end{align}
Here the subscript now denote the solution $\psi$ evaluated at $n$th timestep. Now that the time derivative is handled, at what timestep do we evaluate the $F(\psi)$ term? We could evaluate it at $\psi_{n}$ or $\psi_{n+1}$ or maybe other possible previous times. Let us consider evaluating at $\psi_{n}$. Now plugging everything in we obtain the following scheme, known as the forward Euler scheme \cite{durran}
\begin{align}
\psi_{n+1} = \psi_{n} - \Delta t F(\psi_{n}).
\end{align} 
Here the right hand side is a function solely of $\psi_{n}$. When scheme can be written as a function of all previous terms, the scheme is known as explicit. However, we could have chosen to evaluate $\psi$ at $\psi_{n+1}$ and we would obtain the backward Euler scheme \cite{durran} 
\begin{align}
\psi_{n+1} = \psi_{n} - \Delta t F(\psi_{n+1}).
\end{align} 
Here, if $F(\psi)$ is nonlinear function, we cannot isolate for $\psi_{n+1}$. Instead we would have to use some sort of root finding solver to find the correct $\psi_{n+1}$ given $\psi_{n}$. Schemes that exhibit this property are known as implicity schemes. Depending on the problem at hand, one scheme may be better than the other. For this thesis, we uses Adams-Bashforth 2nd and 3rd order, follwing previous work \cite{bc2000c,waitesmol2008}.

Once we have choosen a time-stepping scheme, the stability of the scheme is important to know. If we pick $\Delta t$ too large, for example, the right handside might grow too large and become unstable. By finding the stability region of a scheme, it tells us when we can except our solution to not blow up. To investigate this we define the amplitude A to be
\begin{align}
A = \frac{\psi_{n+1}}{\psi_{n}},
\end{align}
and stability is defined to be 
\begin{align}
|A|^{2} \le 1,
\end{align}
since $A$ can potentially be complex. The reasoning from this equation arises from rearranging the above so that $\phi_{n+1}=A\phi_{n}$. If $|A|\le 1$ then the solution is not growing in time. If $|A|>1$ then at each time-step the solution is being multiplied by a number greater then one, hence it is growing.  To illustrate, we use the second order Adams-Bashforth scheme \cite{durran}
\begin{align}
\psi_{n+1} = \psi_{n} + \frac{\dt}{2}(3F(\psi_{n}) - F(\psi_{n-1})).
\end{align} 
Clearly the growth rate of a scheme will depend on the given function $F(\psi)$. However the standard procedure is to investigate the following model problem \cite{durran}
\begin{align}
\frac{d\psi}{dt} =  (\lambda + i\omega)\psi.
\end{align}
where $\lambda,\omega$ are real numbers. 

To determine the stability region we use the method from Leveque \cite{leveque} which gives a stability region of
\begin{align}
z = \frac{2(\xi^{2}-\xi)}{3\xi -1}
\end{align}
where $\xi$ is the unit circle. Fig.~\ref{ab_stab} illustrates the stability region of the Adams-Bashforth second order.
\begin{figure}
\begin{center}
\includegraphics[width=\textwidth]{ab_stab}
\caption{Stability region of Adams-Bashforth second order. Adapted from \cite{trefethen_spectral}}
\label{ab_stab}
\end{center}
\end{figure}

To illustrate time-stepping schemes and how they are applied to problems, consider the following one dimensional advection equation\cite{trefethen_spectral}
\begin{align}
\frac{\partial u}{\partial t} + c(x)\frac{\partial u}{\partial x} = 0,\qquad c(x)=\frac{1}{5}+\sin^{2}(x-1), \qquad u(x,0)=e^{-100(x-1)^{2}}, x\in[0,2\pi], t>0
\end{align}
where we are solving on a periodic domain. The physical interpretation of this equation is the simple one dimensional advection of a velocity field $u(x,t)$ by the fixed field $c(x)$. To evaluate the advection term, we use a spectral method with 2/3s dealiasing. As discussed in the previous section, there is still non-standard terminology throughout the literature regrading the names of the various methods of using FFTs. Trefethen, for example, calls pseudospectral methods, `collocation methods'\cite{trefethen_spectral} which is also a term used by Durran \cite{durran}, although Durran prefers the term pseudospectral.  

For a time-stepping scheme, we use a second-order Adams-Bashforth\cite{durran} scheme. Re-writing wave-equation as
\begin{align}
\frac{\partial u}{\partial t} = -c(x)\frac{\partial u}{\partial x} = F(u)
\end{align}
which we can write in Fourier space as
\begin{align}
\frac{\partial \hat{u}}{\partial t} = \widehat{-c(x)\frac{\partial u}{\partial x}} = F(\hat{u})
\end{align}
the Adams-Bashforth scheme is
\begin{align}
\hat{u}^{n+1} = \hat{u}^{n} + \frac{\dt}{2}[3F(\hat{u}^{n})-F(\hat{u}^{n-1})]
\end{align}
Since the Adams-Bashforth scheme uses a previous time-step, we will use forward Euler for the first time-step. To evaluate $F(u)$ we will use the spectral differentiation, as discussed above. See Appendix A for the sample MATLAB code which illustrates how to achieve dealiasing in practice. 

\begin{figure}
\begin{center}
\includegraphics[width=\textwidth]{spectral_example.pdf}
\caption{Spectral solution to the advection equation with variable windspeed. Figure is a reproduced version of Output 6 of Trefethen \cite{trefethen_spectral}}
\label{spectral_example}
\end{center}
\end{figure}
Fig.~\ref{spectral_example} demonstrates the solution to the above advection equation using spectral methods. As can be observed, the solution is very smooth and evolves very clearly for only $N=128$.

In the previous section we discussed spectral and pseudospectral schemes. To illustrate the differences between the two, we solve the viscous Burgers equation using both spectral and pseudospectral methods \cite{durran}
\begin{align}
\frac{\partial \psi}{\partial t} + \psi\frac{\partial \psi}{\partial x} = \nu \frac{\partial^{2}\psi}{\partial x^{2}}.
\end{align}
The initial condition used was a Gaussian. To handle the time-stepping, we use an Adams-Bashforth second order as discussed above.
\begin{figure}
\begin{center}
\includegraphics[width=\textwidth]{spec_vs_pspec}
\caption{Solution to the viscous Burgers equation using spectral (red) and pseudospectral (blue).}
\label{spectralvpseudo}
\end{center}
\end{figure}
Fig.~\ref{spectralvpseudo} illustrates the results of the numerical experiment. Initially, both the spectral and pseudospectral codes have similar energy wise but after around $1$ time unit, the pseudospectral is dissipating energy faster than the spectral code. However, the key difference is that at around $T=3.6$ the pseudospectral code diverges due to aliasing error. The spectral code, which does dealias, does not experience this divergence and continues to decay due to viscous effects. Indeed, it can be shown \cite{durran} that for inviscid problems, the spectral method will conserve energy. This example illustrates the importance and necessity of dealiasing. 

%Discuss dealiasing of linear and nonlinear code as well. Show plots.  
\section{Navier-Stokes in Fourier Space}
\subsection{Fourier Transformed Navier-Stokes}
We now turn to the formulation of the Navier-Stokes equations in the Fourier domain which provides a convenient formulation to analyse the underlying mechanisms of turbulence. Recall that in the Fourier domain, derivatives become multiplication of wavenumbers which converts the spatial parts of the Navier-Stokes equations into algebraic equations. To demonstrate the formulation in Fourier space, let us cast the standard Navier-Stokes equations into Fourier space. 
\begin{align}
\frac{\partial \bm{u}}{\partial t} + \bm{u}\cdot\nabla\bm{u} = -\frac{1}{\rho_{0}}\nabla p + \nu\nabla^{2}\bm{u}, \qquad \nabla\cdot\bm{u}=0
\end{align}
Taking the Fourier transform of the above equation is straight-forward for all terms except the advection term $\bm{u}\cdot\nabla\bm{u}$, which we postpone for now.

In Fourier space, we can also exploit the following observation to eliminate the pressure term. The incompressibility condition becomes $\textbf{k}\cdot\hat{\bm{u}}(\textbf{k},t)=0$ in Fourier space. Geometrically, this means that the vectors $\textbf{k}$ and $\hat{\bm{u}}$ are orthogonal.  
%To see this define a $\textbf{k}$-plane and a $\hat{\bm{u}}$-plane. The defining equation of a plane is $ax+by+cz=d$ with the normal vector $\textbf{n}=(a,b,c)$. Here the normal vectors are $\textbf{k}$ and $\hat{\bm{u}}$. Thus if the normal vectors are orthogonal, the planes are orthogonal.  
This realisation tells us that vectors that are proportional to \uhatm are orthogonal to vectors that are proportional to \kvecm. Thus writing out the Navier-Stokes equations
\begin{align}
\frac{\partial \uhat}{\partial t} + \mathcal{F}(\bm{u}\cdot\nabla\bm{u}) = -\frac{1}{\rho_{0}} \kvec\hat{p} - \nu k^{2}\uhat\label{NS_fourier_1},
\end{align}
take the dot product with \kvecm and using the orthogonality condition we obtain
\begin{align}
\kvec\cdot\mathcal{F}(\bm{u}\cdot\nabla\bm{u}) + \frac{1}{\rho_{0}}k^{2}\hat{p}=0\label{pressure_fourier_1}.
\end{align}
Isolating for pressure and substituting back into (\ref{NS_fourier_1}) we obtain
\begin{align}
\frac{\partial \uhat}{\partial t} + \mathcal{F}(\bm{u}\cdot\nabla\bm{u})(\textbf{1}-\frac{\kvec\kvec}{k^{2}})= - \nu k^{2}\uhat\label{NS_fourier_2}.
\end{align}
This result is unsurprising, since all we have done is take the divergence of the Navier-Stokes equations, which in Fourier space corresponds to taking the dot product with respect to $\kvec$. But using this observation we can avoid the need for solving the pressure altogether because the pressure term is orthogonal to the $\uhat$-plane. But what about the advection term? As can be seen in (\ref{NS_fourier_2}), it has this factor $\textbf{1} - \kvec\kvec/k^{2}$ multiplying it. This term represents a projection into the $\uhat$-plane. The advection term can thought of a vector that is pointing in some direction in-between the planes of $\kvec$ and $\uhat$. By projecting the advection term into the $\uhat$-plane, we would have a set of equations that are independent of the pressure completely.

\subsection{Projection Tensor}
In order to project the Navier-Stokes equations onto the $\uhat$-plane, we define the following projection operator
\begin{align}
\textbf{P}=\textbf{1} - \frac{\kvec\kvec}{k^{2}} \text{ or } P_{ij}(\kvec) = \delta_{ij} - \frac{k_{i}k_{j}}{k^{2}}
\end{align}
where we are using Einstein summation notation \cite{lesieur,wald}. It is straightforward to verify that $P_{ij}P_{jk}=P_{ik}$ or in matrix notation $\textbf{P}^{2}=\textbf{P}$, in other words the projection tensor is idempotent. Idempotence is a defining feature of projection operators \cite{MeyerLinAlg}. It is straightforward to verify that $k_{j}P_{ij}=0$ and $\hat{u}_{j}P_{ij}=\hat{u}_{i}$. These simple observations confirm that the projection tensor projects a vector onto the $\uhat$-plane. 

Applying $P_{ij}$ to (\ref{NS_fourier_1}) we obtain the following 
\begin{align}
\frac{\partial \uhat}{\partial t} + \textbf{P}\mathcal{F}(\bm{u}\cdot\nabla\bm{u}) =  -\nu k^{2}\uhat\label{NS_fourier_3}
\end{align}
where $\textbf{P}$ is acting on the Fourier transform of the advection term. In order to compute the Fourier transform of the advection term, we note that 
\begin{align}
\bm{u}\cdot\nabla\bm{u} = u_{j}\frac{\partial u_{i}}{\partial x_{j}} = \frac{\partial (u_{i}u_{j})}{\partial x_{j}}
\end{align}
where the incompressibility condition has been used to bring the velocity inside the derivative. Thus we are able to write
\begin{align}
\mathcal{F}(\bm{u}\cdot\nabla\bm{u})=ik_{j}\int_{\textbf{p}+\textbf{q}=\kvec}d\kvec\hat{u}_{i}(\textbf{p})\hat{u}_{j}(\textbf{q})
\end{align}
and hence we can finally write out the Navier-Stokes equations in Fourier space as \cite{lesieur}
\begin{align}
\frac{\partial \hat{u}_{i}}{\partial t} + iP_{ij}k_{m}\int_{\textbf{p}+\textbf{q}=\kvec}d\kvec\hat{u}_{j}(\textbf{p})\hat{u}_{m}(\textbf{q})=  -k^{2}\hat{u}_{i}\label{NS_fourier_3}
\end{align}

\subsection{Numerical Formulation of NS in Fourier}
Although we have eliminated the pressure completely, we still have the nonlinear term in the equation. To formulate this problem numerically, we make the following observation that is useful in spectral methods \cite{lesieur,orszag1972}.

Recall the following identity \cite{kundu,acheson_fluid}
\begin{align}
\bm{u}\cdot\nabla\bm{u} = \bm{\omega}\times \bm{u} - \frac{1}{2}\nabla \bm{u}^{2}
\end{align}
When we apply the projection operator $\textbf{P}$ to the above equation, the $\nabla \bm{u}^{2}$ term will vanish since it is orthogonal to the $\uhat$-plane. For the cross product between the vorticity and velocity, we use the methods discussed above from dealiasing. Thus to evaluate the cross product term, we assume we have the Fourier transform of the vorticity and velocity $\hat{\bm{\omega}},\uhat$ and re-write the cross product term as
\begin{align}
\mathcal{F}(\bm{\omega}\times \bm{u}) = \mathcal{F}(\mathcal{F}^{-1}(\hat{\bm{\omega}})\times\mathcal{F}^{-1}(\uhat))
\end{align}
Using this result we can reformulate the Navier-Stokes equations into a form to be solved numerically using a spectral method
\begin{align}
\frac{\partial \uhat}{\partial t} = \textbf{P}(\kvec)\mathcal{F}(\mathcal{F}^{-1}(\hat{\bm{\omega}})\times\mathcal{F}^{-1}(\uhat))-k^{2}\uhat
\end{align}
where the $\mathcal{F},\mathcal{F}^{-1}$ can be evaluated by FFTs, as discussed above.

This reformulation of the Navier-Stokes equations into Fourier space simplifies numerical calculations immensely and provides many advantages over the real space formulation. The absence of the pressure term means that there is no Poisson equation to be solved at each time-step for the pressure. If one did want the pressure, one can solve (\ref{pressure_fourier_1}) for $\hat{p}$. In addition there is no need to enforce a divergence free solution\footnote{Except possibly at the initial time step and periodically during a simulation due to numerical errors.} as the equations are formulated by definition to satisfy divergence free condition. The only additional technical difficulty is evaluating the vorticity, but this can easily be handled because of the simple structure of the curl. 

The above has been focused on the Navier-Stokes equations and not the Boussinesq equations, however the extension to these equations is straightforward. The only additional complexities are an additional equation which can be solved analogously to the momentum equation \cite{lesieur}.
\subsection{Integrating Factor}
Another advantage of the Fourier formulation is the ability to exactly integrate the diffusion term. Let us denote the advective projective term as $F(\hat{\bm{u}})$ and we have
\begin{align}
\frac{\partial \uhat}{\partial t} + \nu k^{2}\uhat = F(\uhat) 
\end{align}
where the left-hand side has been explicitly written out. Written in this form, the common trick of writing a product as a derivative is observed since
\begin{align}
 \frac{\partial \uhat}{\partial t} + \nu k^{2}\uhat= e^{-\nu k^{2}t}\frac{\partial }{\partial t}(\uhat e^{\nu k^{2}t})
\end{align}
Thus we can re-write the Navier-Stokes equations as 
\begin{align}
\frac{\partial }{\partial t}(\uhat e^{\nu k^{2}t})= e^{\nu k^{2}t}F(\uhat) 
\end{align}
For notational convenience, let us write $g(t) = e^{\nu k^{2}t}$ and the note the following trivial identities
\begin{align}
g(t\pm\dt) = g(t)g(\pm\dt),\qquad g(0) = 1, \qquad g(t)^{-1} = g(-t).\label{int_fact_ident}
\end{align}
With this notation the Navier-Stokes equations become
\begin{align}
\frac{\partial (\uhat g(t))}{\partial t} = g(t)F(\uhat).
\end{align}
 Now let us solve the above system using an Adams-Bashforth 2nd order time-stepping scheme. Initially we obtain
\begin{align}
\uhat^{n+1}g(t_{n}+\dt) = \uhat^{n}g(t_{n}) + \frac{3}{2}\dt g(t_{n})F(\uhat^{n}) - \frac{1}{2}\dt g(t_{n-1})F(\uhat^{n-1}).
\end{align}
Using the identities in (\ref{int_fact_ident}) the scheme reduces to
\begin{align}
\uhat^{n+1} = g(-\dt)\uhat^{n} + \frac{3}{2}\dt g(-\dt)F(\uhat^{n}) - \frac{1}{2}\dt g(-2\dt)F(\uhat^{n-1}),
\end{align}
and the diffusion term has been reduced to a multiplication by a constant factor $g(-\alpha\dt)$.

\subsection{Hyperviscosity}
Hyperviscosity is a method of simulating higher Reynolds number flow by replacing the diffusion term with higher derivatives. In the Fourier picture, the diffusion term is $-\nu k^{2}\uhat$. The diffusion timescale $\tau_{d}$ is given by the inverse of $\nu k^{2}$. This implies that the longest wavelengths (smallest $k$) have very long diffusive time-scales while the shortest wavelengths (larger $k$) have very short diffusive time-scales. This picture makes physical sense since viscosity plays the most important role at the very small scales. As we decrease the viscosity $\nu$ the time-scales of all scales increases. Numerically, if we decrease the viscosity too much, resolution of the smallest scales becomes critical and can lead to unwanted grid-scale effects. Thus, in order to decrease the diffusive time-scales of all wavelengths we can instead vary the power of the wave number. For example, going from $k^{2}$ to $k^{4}$, and adjusting the coefficient accordingly, the time-scales of the various wavelengths would decrease. This is illustrated in Fig~\ref{hyper_vis_example}.

\begin{figure}
\begin{center}
\includegraphics[width=\textwidth]{hyper_vis_example.pdf}
\caption{The inverse diffusion times, $1/\tau_{d}$, of the wavenumbers for the regular viscosity, blue, and the hyperviscosity case, red. The hyperviscosity inverse diffusion times are lower then the regular viscosity case, corresponding to longer diffusion times. This has the effect of simulating larger Reynolds numbers. Note that at $k_{max}$ the two cases coincide.}
\label{hyper_vis_example}
\end{center}
\end{figure}

REPLACE THIS PARAGRAPH WITH EXAMPLES FROM LITERATURE
%In general we consider $k$ to be an even number, since this corresponds to even derivatives, however as long as we correctly choose the co-efficient in front of the $k^{n}$ to be $-1$, we can obtain different degrees of diffusion. But if we allow this, odd powers of $k$ no longer correspond to odd derivatives as we will be off by a factor of $i$ and we cannot interpret these derivatives as dispersive terms.

In order to encorporate hyperviscosity numerically, we scale the coefficient by the maximum wavenumber $k_{max}$. We want the diffusive timescales of the smallest scales to be the same for both the regular viscosity and the hyperviscosity. That is we want 
\begin{align}
\nu k_{max}^{2} = \nu_{i}k_{max}^{i},
\end{align}
where $i$ is an even integer. Solving for $\nu_{i}$ then gives the following replacement
\begin{align}
\nu k^{2} \Rightarrow \nu k_{max}^{2-i}k^{i}.
\end{align}
Throughout this thesis, we will use $i=4$. 

\subsection{Concluding Remarks}
Although we have done all our manipulations in Fourier space, we could have equally well formulated the above equations in real space. The idea of projecting the velocity field onto the $\uhat$-plane is motivated by the Helmholtz decomposition. This decomposition states, for any $C^{2}$ vector field in a bounded region of $\mathbb{R}^{3}$, we can decompose the vector field into a divergence-free and curl-free component. The divergence free part corresponds to taking the divergence of the Navier-Stokes equations which would yield the Poisson equation for the pressure. Taking the curl of the Navier-Stokes equation - which we define to be the vorticity - would yield the vorticity equation which does not have a pressure term.


\section{Numerical Scheme for Linear and Nonlinear Boussinesq Equations}
\subsection{Linear Boussinesq Equations Scheme}
We now write the fields as a basic state, denoted by a subscript $0$, plus perturbations, denoted by $\sim$. Ignoring the viscous diffusion of the basic state \cite{drazinreid} and neglecting products of the perturbations, we obtain the following set of linear equations for the perturbations
\begin{align}
\frac{\partial \tilde{\bm{u}}}{\partial t} + \omega_{z0}\hat{\bm{e}}_{z}\times \tilde{\bm{u}}+\tilde{\boldsymbol{\omega}}\times \bm{u}_{h0} = -\nabla(\tilde{p}+\bm{u}_{h0} \cdot \tilde{\bm{u}}) - \tilde{\rho}'\hat{\bm{e}}_{z} + \frac{1}{Re}\nabla^{2}\tilde{\bm{u}},\label{nsl1}\\
\nabla\cdot\tilde{\bm{u}}=0,\\
\frac{\partial \tilde{\rho}'}{\partial t} + \bm{u}_{h0}\cdot \nabla_{h}\tilde{\rho}'-\frac{1}{F_{h}^{2}}\tilde{w} = \frac{1}{ScRe}\nabla^{2}\tilde{\rho}',\label{nsl3}
\end{align}
where $\tilde{\boldsymbol{\omega}}=\nabla \times \tilde{\bm{u}}$. To test how much the basic state diffuses, we used the full nonlinear code, which will be discussed below, with the same Lamb-Chaplygin dipole initial state, and let it diffuse out. We found that over $50$ time units, the basic state only diffused by about $10\%$, and hence we are justified in ignoring the viscous diffusion of the basic state. 

As stated above, the Lamb-Chaplygin dipole is oriented vertically. As a result we can separate the perturbation into the vertical and horizontal directions as 
\begin{align} 
[\tilde{\bm{u}},\tilde{p},\tilde{\rho}'](x,y,z,t) = [\bm{u},p,\rho'](x,y,t)e^{ik_{z}z} + \text{c.c.},
\end{align}
where c.c. is the complex conjugate. From here we can now take the 2D Fourier transform and and recall the projection operator $\textbf{P}(\textbf{k})$, with components $P_{ij}(\textbf{k})=\delta_{ij} - k_{i}k_{j}/k^{2}$ to eliminate pressure, to obtain a set of equations for the Fourier coefficients 
\begin{align}
\frac{\partial \hat{\bm{u}}}{\partial t} = \textbf{P}(\textbf{k})[\widehat{\bm{u}\times \omega_{z0}\hat{\bm{e}}_{z}} + \widehat{\bm{u}_{h0}\times\bm{\omega}}-\hat{\rho}'\hat{\bm{e}}_{z}] - \frac{k^{2}}{Re}\hat{\bm{u}},\label{solve1}\\
\frac{\partial\hat{\rho}'}{\partial t} = -i\bm{k}_{h}\cdot\widehat{\bm{u}_{h0}\rho'} + \frac{1}{F_{h}^{2}}\hat{w}- \frac{k^{2}}{ScRe}\hat{\rho}',\label{solve2}
\end{align}
where $k_{z},Re,Sc,F_{h}$ are input parameters, $\bm{k}_{h}=(k_{x},k_{y})$ is the horizontal wavenumber and $k^{2}=k_{x}^{2}+k_{y}^{2}+k_{z}^{2}$ is the total wavenumber. 

To numerically solve (\ref{solve1}) and (\ref{solve2}), we use a spectral transform method to evaluate derivatives, with 2/3-rule de-aliasing and second order Adams-Bashforth for time-stepping. Each simulation was initialised with a random field and integrated over an $N\times N$ grid for 100 time units to determine the behaviour of the fastest growing mode. After several time units, the leading eigenmodes for $\bm{u},\rho$ behave exponentially (e.g. Billant and Chomaz \cite{bc2000c})
\begin{align}
\bm{u},\rho \propto C(x,y)e^{\sigma t},
\end{align}
and we can obtain the largest growth rate by the formula
\begin{align}
\sigma = \lim_{t\rightarrow\infty}\frac{1}{2}\frac{d\ln E}{dt}\label{sigma1},
\end{align}
where $\sigma$ is the real growth rate of the mode and $E$ is the kinetic energy $\frac{1}{2}(u^{2}+v^{2}+w^{2})$. This follows directly from the exponential behaviour of the leading eigenmode. The energy behaves as
\begin{align}
E \sim \frac{3}{2}C(x,y)^{2}e^{2\sigma t} \Rightarrow \ln E = \ln(3C^{2}/2) + 2\sigma t
\end{align}
and upon taking the time derivative of both sides yields the desired result. To obtain the energy, we use Parseval's theorem which relates the real space energy to the sum of the squares of the Fourier coefficients. Thus the total kinetic energy is obtained by summing the squares of the Fourier transform of the velocities over all wavenumbers. 

To evaluate $\sigma$, we compute the average value of the growth rate from the time series. Initially, there is transient behaviour where the various modes are all growing and none dominant. After $t=20$, after the initial transient behaviour has died out and the leading mode dominants in all the cases observed. In the case of an oscillatory growth rate, as considered in \cite{bc1999}, we drop the assumption that $\sigma$ is real and instead compute the growth rate from
\begin{align}
\sigma_{r} = \lim_{t\rightarrow \infty} \frac{1}{2T}\ln\left(\frac{E(t+T)}{E(t)}\right)\label{sigma2},
\end{align}
where $T$ is the period of the oscillatory mode. The imaginary growth rate is given as $\sigma_{i}=2\pi/T$. As above, we compute $\sigma$ from the time series beginning at $t=20$, however we first measure the period $T$ from roughly 10 oscillations, and then compute the average.  

%To simulate higher Reynolds number, we use a hyperviscosity operator. The $\nu\nabla^{2}$ diffusion term is replaced with a $\nu_{4}\nabla^{4}$ diffusion term. The $\nu_{4}$ coefficient is chosen so that $\nu k_{max}^{2} = \nu_{4}k_{max}^{4}$, where $k$ is the maximum dealiased horizontal wave number. This allows us to define the hyperviscosity Reynolds number $Re_{h}=Re k_{max}^{2}$. The hyperviscosity simulation was run with $F_{h}=0.1$ and $Re=20000$ with the same numerical parameters as the regular viscosity simulation.

\subsection{Nonlinear Boussinesq Scheme}

We now turn to the problem of numerically solving the nonlinear Boussinesq equations. Discuss KE transfer + describe how the code works.

%-------------------------------------------------------------------------------
%
%   LINEAR THEORY 
%
%-------------------------------------------------------------------------------
%\chapter{Linear Simulations}

\section{Introduction}
In this chapter we extend the results of Billant and Chomaz \cite{bc2000c} to the sub-buoyancy length-scales. We also reproduce the numerical results presented by Billant and Chomaz and confirm their conclusions about the existence of the zigzag instability at the buoyancy scale. The results in this chapter are based on the paper by Bovard and Waite \cite{bovard2013}.

\section{Set-up}
The equations that we want to solve are the linear Boussinesq equations, which we derived in Chapter 2 and for which we derived the numerical scheme in Chapter 3. We repeat the equations here:
\begin{align}
\frac{\partial \hat{\bm{u}}}{\partial t} = \textbf{P}(\textbf{k})[\widehat{\bm{u}\times \omega_{z0}\hat{\bm{e}}_{z}} + \widehat{\bm{u}_{h0}\times\bm{\omega}}-\hat{\rho}'\hat{\bm{e}}_{z}] - \frac{k^{2}}{Re}\hat{\bm{u}},\\
\frac{\partial\hat{\rho}'}{\partial t} = -i\bm{k}_{h}\cdot\widehat{\bm{u}_{h0}\rho'} + \frac{1}{F_{h}^{2}}\hat{w}- \frac{k^{2}}{ScRe}\hat{\rho}'.
\end{align} 
The parameter regime to be explored is a function of $k_{z},Re,F_{h}$. 

In our simulations, we investigate a range of Reynolds numbers $Re=5000-20{,}000$ and Froude numbers $F_{h}=0.05-0.2$. These numbers have been chosen to be within the range of the results of Billant and Chomaz \cite{bc2000c}, who investigated a similar range. This range also falls within potential experimental regimes which could be investigated using the techniques of Billant and Chomaz \cite{bc2000a}. For each $F_{h}$ and $Re$ a wide range of vertical wavenumber, $k_{z}$, were investigated with $k_{z}$ ranging from $1$ to $200$. This wavenumber range incorporates the scale of the zigzag instability down to the viscous damping scale. We additionally consider the Schmidt number to be unity, following the main results of Billant and Chomaz \cite{bc2000c}. In order to compare with experiments, Billant and Chomaz did investigate a non-zero Schmidt number, however their numerical work was dedicated to $Sc=1$ and thus we do not consider different Schmidt numbers here. 

%For certain $Re$ and $F_{h}$, as we shall see below, the viscous damping range changes and thus when the simlations were clearly in the viscous dissipation range, no further simulations were conducted.  We additionally consider the Schmidt number to be unity which is a standard practice (cite... justify? numerically difficult?). 

For our simulations a box size of $L=9$ with $N=512$ grid points was used. The reason for choosing $L=9$ was to minimise interactions of the fields with themselves across the periodic boundary. Lengths of size $L=5$ have been used in practice \cite{augier2012}, however we erred on the side of caution and repeated the box size conditions of Billant and Chomaz \cite{bc2000c}. For robustness results were checked at smaller box size and agree with the tests of \cite{bc2000c}. We used the timesteps of $\Delta t=0.000950$ for $F_{h}=0.2,Re=2000,5000,10000$ and $\Delta t=0.000375$ for all the other simulations. These were chosen following Billant and Chomaz \cite{bc2000c}. 


The code was written in Fortran and all the FFTs were done by FFTW V3.0\cite{fftw} and was tested by comparing the growth rates given by Billant and Chomaz in \cite{bc2000c}, specifically their Figure  1 and Tables 1 and 2. Unlike Billant and Chomaz \cite{bc2000c} we did not restart each simulation with the previous eigenmode because we used a parallel approach for evaluating multiple $k_{z}$ simultaneously. The trade-off is the individual simulations had to be run longer but they could be done in parallel allowing for a greater variability in the parameters explored. 



\section{Growth Rate}
As discussed in Chapter 3, the growth rates are computed by evaluating the time derivative of the energy of the eigenmode. Fig.~\ref{sigma_examples} demonstrates two different types of growth rates that are representative of all the growth rate plots for all the simulations. Here we have chosen $F_{h}=0.1, Re=20{,}000$ and $k_{z}=20$ in panel a and $k_{z}=60$ in panel b. In both panels, the perturbation initially goes through some transient behaviour which is when the various eigenmodes all grow together and none of have become dominant. After about 20 time units this transient behaviour has died off the dominant eigenmode has emerged. This 20 time unit delay was observed for all simulations. In panel (a), there is a case of an oscillatory growth rate. As can be observed, the growth rate of this wavenumber is highly oscillatory and thus the true value of the growth rate must be computed using the period of the oscillatory. Panel (b) represents the more typical case of the growth being non-oscillatory and quickly settling down onto a certain value. 
\begin{figure}
\begin{center}
\includegraphics[width=\textwidth]{sigma_examples.pdf}
\caption{Time series of the growth rate, obtained from the derivative of the energy, for $F_{h}=0.1$ and $Re=10{,}000$. Panel (a) is $k_{z}=20$ and Panel (b) is $k_{z}=60$.}
\label{sigma_examples}
\end{center}
\end{figure}

Using these types of plots of the growth rate, we are able to extract the leading growth rate of the maximum eigenmode by examining the long term behaviour. From these times series, we determined the maximum growth rate of the leading eigenmode for a wide range of wavenumbers, Reynolds numbers, and Froude numbers. The growth rate curves for a given $F_{h}$ and $Re$ can be plotted by determining this maximum growth rate for each vertical wavenumber. Following Billant and Chomaz\cite{bc2000c}, we scale the vertical wavenumber by $F_{h}$ to obtain a horizontal scaling $k_{z}F_{h}$.  This is, in our units, scaling by the buoyancy scale $L_{b} = U/N$. Recall that $k_{z}$ has been non-dimensionalised by $R$, i.e. $k_{z}= k_{z}'R$ where $k_{z}'$ is the dimensional wavenumber with units of inverse length. So the dimensionless $F_{h}k_{z}$ would become $F_{h}k'_{z}R$ in dimensional units, $k'_{z}$ denoting the dimensional vertical wavenumber. Re-writing this we obtain $F_{h}k'_{z}R=k_{z}'U/N=k_{z}'L_{b}$, using the definition of the Froude number, and hence dimensional $F_{h}k_{z}'R=k_{z}'L_{b}$ but since $R=1$, $k_{z}'$ and $k_{z}$ have the same value, but with different units. Thus the dimensionless scaling $F_{h}k_{z}$ is like scaling by the buoyancy length. 

\begin{figure}
\begin{center}
\includegraphics[width=\textwidth]{fixed_froude_varying_reynolds}
\caption{Growth rate $\sigma$ as a function of $k_{z}F_{h}$ for fixed $F_{h}=$(a) $0.2$, (b) $0.1$, (c) $0.05$ with Re$=2000$ (cyan), Re$=5000$ (red), Re$=10000$ (black), Re$=20{,}000$ (blue). In panel (b) the green line is the hyperviscosity case with $Re=20{,}000$.}
\label{FixFhVaryRe}
\end{center}
\end{figure}

Fig.~\ref{FixFhVaryRe} shows the largest eigenmode growth rate as a function of vertical wavenumber for fixed $F_{h}$ and $Re$ for a wide range of $F_{h}$ and $Re$. The qualitative behaviour for the growth rates at different Reynolds numbers are very similar to one another. We now investigate the various regimes in some detail.

At small $F_{h}k_{z}$, the growth rate reaches a local maximum, the zigzag peak, as predicted by Billant and Chomaz \cite{bc2000a,bc2000b,bc2000c}.  As discussed in Chapter 2 the zigzag instability appears at the buoyancy wavelength $L_{b}$. This is clear from the figure where we can see that peak growth rate occurs at the same $F_{h}k_{z}$, here roughly $F_{h}k_{z}\approx 0.6$. In panels (a) and (b) it is especially clear that the growth rate curves all collapse onto each other and have almost identical growth rates regardless of Reynolds number, confirming the results of the prediction of Billant and Chomaz \cite{bc2000b,bc2000c}. Panel (c) has similar growth rates for $Re=20{,}000,10{,}000,$ and $5000$ but for $Re=2000$ the growth rate is a bit lower. Despite the slight difference in growth rates, they do all occur at the same $k_{z}F_{h}$. The $Re=2000$ case being lower suggests that the diffusion effects are beginning to dominate, as can be seen by the rapid decrease in the growth rate of the curve. Since the theoretical results of Billant and Chomaz \cite{bc2000b} were conducted in an inviscid regime, it is not unexpected to observe a breakdown of this assumption. Some numerical experiments were run at even lower Froude number, but here the diffusion was dominating and the growth rate curves were dropping off very rapidly near the zigzag peak and thus even obscuring the zigzag peak. Thus for $F_{h}<0.05$ and $Re<10{,}000$ we find that the sub-buoyancy length scales are being diffused out. 


For $k_{z}$ increasing beyond the zigzag peak, the growth rate then decreases for increasing $F_{h}k_{z}$ to a local minimum before increasing to a second local maximum. At this local minimum oscillatory growth rates are observed. The imaginary part of the growth rate $\sigma_{i}$ remains zero everywhere else except in this small region between the two local maximum, with the exception of very small $k_{z}F_{h}$. Oscillatory growth rates are also observed in this small regime, as observed in \cite{bc1999} and have been observed before \cite{bc1999} and we did not explore this regime. Fig.~\ref{oscillatory_growth} displays the imaginary growth rates for $F_{h}=0.05$ and $Re=10{,}000, 20{,}000$. The imaginary growth rate for $Re=10000$ is a straight line but such a trend is not observed for $Re=20{,}000$. The shape of this curve, when it did appear, depended highly on the Reynold and Froude number, and did not suggest any sort of general result that could be determined about the imaginary growth rate. Additionally, the range of this oscillatory instability also depended on the Froude and Reynolds number and also did not suggest any sort of general result. Due to these observations, we did not explore this regime too closely. 
\begin{figure}
\begin{center}
\includegraphics[scale=0.65]{oscillatory_growth}
\caption{Imaginary part of the growth rate for $F_{h}=0.05$ and $Re=20{,}000$ (blue) and $Re=10{,}000$ (black).}
\label{oscillatory_growth}
\end{center}
\end{figure} 

After this local minimum, the growth rate increases to a secondary maximum. We will discuss it further below. Continuing to even smaller vertical scales, viscous effects increase and may damp out the instability, and hence the growth rate decays with increasing $k_{z}F_{h}$ in the limit of large $k_{z}F_{h}$. 

For $F_{h}=0.2$ (Fig.~\ref{FixFhVaryRe}a), the peak growth rate of the short-wave instability exceeds that of the zigzag instability for increasing Reynolds numbers. The growth rates at the second peak is smaller for $F_{h}=0.1$ (Fig.~\ref{FixFhVaryRe}b), but they continue to increase with increasing $Re$. For $F_{h}=0.05$ (Fig.~\ref{FixFhVaryRe}c), the second peak is weaker than the zigzag peak. Fig.~\ref{FixReVaryFh} shows the growth rate for fixed Reynolds numbers with varying Froude numbers. Examining the case of $Re=20{,}000$ (Fig.~\ref{FixReVaryFh}a), the growth rate at the second peak increases with increasing Froude. A similar result is observed for $Re=10000$ and $5000$ (Fig.~\ref{FixReVaryFh}b-c). $Re=2000$ is not included because viscous effects have damped out the second peak in this case. Overall, the dependence of the short-wave growth rate on Froude is also more pronounced than that of Reynolds. For example, the growth rate of the second peak at fixed $Re=20{,}000$ (Fig.~\ref{FixReVaryFh}a) doubles from $F_{h}=0.05$ to $F_{h}=0.2$. By contrast, at fixed $F_{h}=0.2$ (Fig.~\ref{FixFhVaryRe}a), the increase in the growth rate from $Re=5000$ to $Re=20{,}000$ is only about $25\%$ larger. 

\begin{figure}
\begin{center}
\includegraphics[width=\textwidth]{fixed_reynolds_varying_froude}
\caption{Growth rate $\sigma$ as a function of $k_{z}F_{h}$ for fixed $\text{Re}=(a) 20{,}000, (b) 10000, (c) 5000$ with $F_{h}=0.05$ (red), $F_{h}=0.1$ (black), $F_{h}=0.2$ (blue).}
\label{FixReVaryFh}
\end{center}
\end{figure}
The above analysis demonstrates that the short-wave growth-rate peak moves to larger $k_{z}F_{h}$ with increasing $F_{h}$ and increasing $Re$, but has a stronger dependence on Froude than Reynolds. Some of this joint dependence can be explained by examining the dependence on the buoyancy Reynolds number $Re_{b}=F_{h}^{2}Re$ \cite{riley2003,hebert2006,brethouwer2007}. In stratified turbulence, the buoyancy Reynolds number is analogous to the Reynolds number in the viscous term due to the vertical gradients \cite{brethouwer2007}. As $k_{z}$ increases, we move to smaller vertical scales where the vertical viscosity terms, controlled by the buoyancy Reynolds number, dominates, so it follows that the second peak may be governed by $Re_{b}$. In Fig.~\ref{Buoy} the location of the second peak from Fig.~\ref{FixFhVaryRe} is plotted as a function of the buoyancy Reynolds number. The peak location line is approximately linear and can be fitted with the curve $k_{z}F_{h}= Re_{b}^{2/5}$, which is plotted. This scaling implies that the vertical wavenumber, $k_{z}$, of the short-wave instability is approximately 
\begin{align}
k_{z} \sim F_{h}^{-1/5} Re^{2/5}\label{buoyscale}.
\end{align} 
The dependence of the growth rate on $k_{z}F_{h}$ appears to be similar in the cases with different $F_{h}$ and $Re$ but the same $Re_{b}$. Fig.~\ref{ReBuoy} demonstrates the similarity of the growth rate plotted against $k_{z}F_{h}$ for two cases with $Re_{b}=500$ and two cases with $Re_{b}=50$. For both cases, the locations of the zigzag and second peak line up quite well. The difference between the red and blue curves at the second peak is $4\%$ for $Re_{b}=200$ and $6\%$ for $Re_{b}=50$, a reasonable variation. 

% It is interesting to note that the red curve, corresponding to $Re=20{,}000$ and $F_{h}=0.1$ (a), $F_{h}=0.05$ (b), is lower then the blue curve, corresponding to $Re=5000$ and $F_{h}=0.2$ (a), $F_{h}=0.1$ (b). This supports the observation, and is clear from the definition of the buoyancy Reynolds number, that the stratification may play a more important role in the instability than the viscosity.   

\begin{figure}
\begin{center}
\includegraphics[scale=0.65]{second_peak_buoyancy}
\caption{The location of the second peak as a function of the buoyancy Reynolds number $Re_{b}$. $k_{z}F_{h}$ is taken from Fig.~\ref{FixFhVaryRe}. The straight line is $Re_{b}^{2/5}$.}
\label{Buoy}
\end{center}
\end{figure}
\begin{figure}
\begin{center}
\includegraphics[scale=0.65]{buoyancy_reynolds}
\caption{Growth rate $\sigma$ as a function of $F_{h}k_{z}$ for fixed $Re_{b}$. In (a), red is $Re=20{,}000, F_{h}=0.1$ and blue is $Re=5000, F_{h}=0.2$, both corresponding to $Re_{b}=500$; in (b) red is $Re=20{,}000, F_{h}=0.05$ and blue is $Re=5000, F_{h}=0.1$, both corresponding to $Re_{b}=50$.}
%\caption{Growth rate $\sigma$ as a function of $k_{z}$ for fixed $Re_{b}=(a) 200, (b) 50$. In (a) red corresponds to $Re=20{,}000, F_{h}=0.1$ blue to $Re=5000, F_{h}=0.2$, in (b) red corresponds to $Re=20{,}000, F_{h}=0.05$, blue $Re=5000, F_{h}=0.1$}
\label{ReBuoy}
\end{center}
\end{figure}


In Fig.~\ref{FixFhVaryRe} (b) the green curve corresponds to a hyperviscosity run with $Re=20{,}000$, which has $Re_{h}=2.8\times 10^{8}$. The motivation for using hyperviscosity is to capture the higher-Reynolds number regime by restricting dissipation to only the largest wavenumbers. As the hyperviscosity run demonstrates, the zigzag peak is independent of Reynolds number and the existence of the peak would be expected at higher Reynolds numbers. For the second peak, we note that the growth rate  of the hyperviscosity run exceeds that of $Re=20{,}000$ for $k_{z}F_{h}>3$ and reaches a maximum around $k_{z}F_{h}=7$. The maximum growth rate in the hyperviscosity case is around $25\%$ larger than the regular viscosity case with $Re=20{,}000$. At $k_{z}F_{h}=12$ we see the hyperviscosity and non-hyperviscosity curves cross. This intersection corresponds to the horizontal wavenumber at which the hyperviscosity damping rate equals the regular viscous damping rate for $Re=20{,}000$. For $k_{z}$ greater than this maximum, the hyperviscosity operator experiences greater damping than the regular viscosity, which can be seen by the sudden drop off of the growth rate. This simulation presents evidence that as $Re\rightarrow \infty$, the growth rate of the second peak will the same order as, or larger than, the growth rate of the zigzag instability. 

\section{Structure} 
Briefly, we discuss the structure of the zigzag instability. Fig.~\ref{zigzag_vorticity} demonstrates the vertical vorticity of the zigzag peak for $F_{h}=0.2,0.1,0.05$ and $Re=20{,}000$. In all vertical vorticity plots that follow, we have normalised the plots so red represents the maximum velocity or vorticity and blue represents the minimum velocity  or vorticity. Since the vorticity field is real, we have chosen the real part. Only $Re=20{,}000$ has been included, see Fig.~\ref{zigzag_fields}, since the structure is nearly identical for $Re=10{,}000$ and $5000$. This confirms the assumption made by Billant and Chomaz \cite{bc2000b} that the zigzag peak emerges at the buoyancy scale $U/N$ independent of $Re$. A detailed study of the structure of the zigzag instability is done by Billant and Chomaz \cite{bc2000c}. 
\begin{figure} 
\begin{center}
\includegraphics[width=\textwidth]{vorticity_zigzag}
\caption{Perturbation vertical vorticity $\omega_{z}$ at the zigzag peak for $F_{h}=0.2 \text{ (left) }, 0.1 \text{ (middle) }, 0.05 \text{ (right) }$ and $Re=20{,}000$.}
\label{zigzag_vorticity}
\end{center}
\end{figure} 
\begin{figure}
\begin{center}
\includegraphics[width=\textwidth]{velocity_fields_kz_6}
\caption{Perturbation fields $u,v,w,\rho$ at the zigzag peak for $F_{h}=0.1$ and $Re=20{,}000$. Here $k_{z}=6$.}
\label{zigzag_fields} 
\end{center}
\end{figure} 

The minimum between the zigzag and short-wave instability is the oscillatory minimum. Fig.~\ref{oscillation_vorticity} and Fig.~\ref{oscillation_fields} are the vorticity and velocity fields respectively. In the vorticity structure, for $Re=20{,}000$ the structure is much more defined then the smoothed out structure of the zigzag instability. At the largest Froude number, the vorticity is in thin strips around the dipole. At the middle Froude number (b),(e),(h) the core of the dipole, new structure has emerged has has a swirl-like pattern. At the smallest Froude number (c),(f),(i), this structure has become much more detailed and a swirl-like pattern has clearly emerged. Additionally, as we increase the viscosity, this swirl-like structure becomes more diffused out. 
\begin{figure}
\begin{center}
\includegraphics[width=\textwidth]{vorticity_oscillation}
\caption{Perturbation vertical vorticity $\omega_{z}$ at local oscillatory minimum for $Re=20{,}000\text{ (top) }, 10000 \text{ (middle) }, 5000 \text{ (bottom) }$; and $F_{h}=0.2 \text{ (left) }, 0.1 \text{ (middle) }, 0.05 \text{ (right) }$.}
\label{oscillation_vorticity}
\end{center}
\end{figure} 
\begin{figure}
\begin{center}
\includegraphics[width=\textwidth]{velocity_fields_kz_20}
\caption{Perturbation fields $u,v,w,\rho$ at the oscillatory minimum for $F_{h}=0.1$ and $Re=20{,}000$. Here $k_{z}=20$.}
\label{oscillation_fields}
\end{center}
\end{figure} 

\begin{figure}
\begin{center}
\includegraphics[width=\textwidth]{vorticity_second_peak}
%\caption{Perturbation vertical vorticity $\omega_{z}$ for $k_{z}$ at the second peak. $F_{h}=(a)(d)(g) 0.2 ,(b)(e)(h) 0.1, (c)(f)(i), 0.05, Re=(a)(b)(c) 20{,}000, (d)(e)(f) 10000, (g)(h)(i) 5000$.}
\caption{Perturbation vertical vorticity $\omega_{z}$ at second peak for $Re=20{,}000\text{ (top) }, 10000 \text{ (middle) }, 5000 \text{ (bottom) }$; and $F_{h}=0.2 \text{ (left) }, 0.1 \text{ (middle) }, 0.05 \text{ (right) }$.}
\label{secondpeak}
\end{center}
\end{figure}
\begin{figure}
\begin{center}
\includegraphics[width=\textwidth]{velocity_fields_kz_60}
\caption{Perturbation fields $u,v,w,\rho$ at the short wave instability for $F_{h}=0.1$ and $Re=20{,}000$. Here $k_{z}=60$.} 
\label{secondpeak_fields}
\end{center}
\end{figure} 
Fig.~\ref{secondpeak} and Fig.~\ref{secondpeak_fields} show the spatial structure of the perturbation vertical vorticity and fields at the second peak for different $Re$ and $F_{h}$. Qualitatively, we observe greater variation for different Froude numbers versus different Reynolds number as suggested above.  At the largest Froude number, the perturbation vorticity is organised in thin strips around and inside the dipole core between the two vortices. Panels (b),(e),(h) have $F_{h}=0.1$ and have a similar overall structure to the larger Froude number. Here, in the cores of the vortices, there is an emergence of a swirl-like pattern. At lower Reynolds number, the structure is spread out due to diffusion, while at higher Reynolds number, small-scale structure is beginning to emerge. This trend continues overall as we move to lower Froude numbers. 

Examining Figs.~\ref{secondpeak} (g)-(i) (fixed $Re$ and decreasing $F_{h}$), the core of the dipoles has a twisting-like behaviour as the Froude number decreases. From this we can conclude that the instability structure of the second peak depends more on the Froude number than on the Reynolds number, which again reinforces the buoyancy Reynolds number scaling.  Indeed, if we consider the cases with $Re_{b}=50$ and $200$ as above, which correspond to Fig.~\ref{secondpeak} (b),(g) and (c),(h) respectively, we can see similar structure in the vorticity fields. Additionally, the anti-symmetric structure of the perturbation can be observed in the dominant eigenmodes in all cases, as found by \cite{bc1999,bc2000c}.

% The vorticity being very thin in the centre is consistent with the results of \cite{pierrehumbert1986} which examined unstratified inviscid vortices at small vertical scales which also demonstrated this behaviour. 
\begin{figure}
\begin{center}
\includegraphics[scale=0.5]{second_peak_vs_zigzag}
\caption{Perturbed vertical vorticity $\omega_{z}$ at (a) the zigzag peak (b) the second peak for $Re=5000, F_{h}=0.2$}
\label{zigzagcomparison}
\end{center}
\end{figure}

Fig.~\ref{zigzagcomparison} shows the perturbation structure for the zigzag peak (a) and the short-wave peak (b) for the case of $Re=5000,F_{h}=0.2$. This case was chosen because the growth rates of the two wavenumbers is roughly the same (see Fig~\ref{FixFhVaryRe} a). The zigzag instability exhibits a quadrupole vorticity structure as discussed in \cite{bc2000c}, which corresponds to a bend and a twist of the basic state dipole. The short-wave instability shares some common overall structure with the zigzag instability. Both have a line of vorticity centred in between two Lamb-Chaplygin vortices and have a ring of vorticity negative vorticity around the outer edges of the dipoles. Additionally, the number of local maximum and minimum remains the same. However, in the short-wave instability, these bands of vorticity have been squeezed into thinner strips and are much more localised along the outer edges of the vortices. In the cores of the dipoles, there is almost no structure and we do not see a quadrupole moment. The full vorticity field of the short-wave instability has a much more dominant twist then the zigzag instability and the bending of the dipole is reduced. As the stratification is increased, this behaviour continues but there is a significant emergence of structure within the cores of the vortices, as observed in Fig~\ref{secondpeak}.


\begin{figure}
\begin{center}
\includegraphics[width=\textwidth]{velocity_field_evolution_u_fh_01_re_20000}
\caption{Structure of the v velocity from $k_{z}=3$ up to $k_{z}=110$ for $F_{h}=0.1$ and $Re=20{,}000$. The wavenumber is increasing left to right and top to bottom. Here the scaling has not been applied to better illustrate the structure.}
\label{evolution}
\end{center}
\end{figure}

Fig.~\ref{evolution} demonstrates the structure of the $v$ component of velocity for $F_{h}=0.1$ and $Re=20{,}000$ for a range of wavenumbers. Here the transition from the zigzag peak, through the oscillatory minimum to the short-wave instability is explicit. Initially the structure has a large positive and negative velocity but as we increase $k_{z}$ this positive velocity is stretched forward and flattened out and becomes stretched out in the middle of the velocity field. The initial negative velocity simply dies out. The initial small areas of positive velocity in front of the dipole undergo an interesting evolution and become larger and stretched out wrapping around the back of the dipole and become very thin. The small initial negative velocity also becomes stretched out but instead forms a v-like shape behind the dipole. The evolution of this wake looks quantitatively similar to the wake of a boat through water. Fig.~\ref{wake} is a loglog plot of the angle of this wake as a function of the perturbation wavelength. The red line is a reference slope of $k=1$ and suggests the relationship between the angle and the wavenumber to be
\begin{align}
\theta \sim \frac{1}{k_{z}}.
\end{align} 
An investigation by Sharman and Wurtele \cite{sharman1983ship} suggests an inverse relationship between the angle of the wake and the vertical wavenumber. However their stratification set-up is different from ours and it is unlikely their analysis carries over directly, however it does suggest there maybe a similar theory that can be derived for the case of waves behind vortices. 

\begin{figure}
\begin{center}
\includegraphics[width=\textwidth]{wake_fh_01_re_20000}
\caption{Angle of the wake behind the v velocity field as a function of the perturbation wavelength for $F_{h}=0.1$ and $Re=20{,}000$. The red line is a reference slope of $1$.}
\label{wake}
\end{center}
\end{figure}

\section{Subdominant modes}
The method we have used so far is limited to only determining the leading eigenmode. An interesting question is whether or not we can determine sub-dominant eigenmodes and what we can learn from these modes. For example, is the same eigenmode dominating at both the zigzag peak and the short wave peak? To investigate this question, we use a Krylov space method to obtain the growth rates for subdominant modes \cite{edwards1994}. Recall that our linear stability problem can be written as 
\begin{align}
\frac{d\bm{u}}{dt} = A\bm{u}
\end{align}
where $\bm{u}=\bm{u}(\bm{x},t)$ is vector of the unknowns and $A$ is the associated linear operator. We can consider solutions of the form $\bm{u}(\bm{x},t)=\hat{\bm{u}}(\bm{x})e^{\sigma t}$, where $\sigma$ is the growth rate. Plugging in we obtain the following eigenvalue equation for $u$
\begin{align}
\sigma \hat{\bm{u}} = A\hat{\bm{u}}.
\end{align}
As mentioned, we have only been trying to find the leading eigenvalue $\sigma$ but since this is an eigenvalue problem, many algorithms exist to find the eigenspectrum. 

If we want to solve this eigenvalue problem, we must discretise the operator $A$. For our numerical scheme, we do not explicitly construct this operator because it would get very large. Our code is equivalent to this matrix multiplication if one were to write all the FFTs as a matrix multiplication. Writing out operators this way in practice is unfeasible since for our problem, it would require a matrix that is of size $4N^{2}\times 4N^{2}$ and is very dense. Thus trying to find the eigenvalues using a direct eigenvalue routine is not possible. Instead we investigate other algorithms that can determine the eigenvalues without computing $A$ explicitly. 

\subsection{Krylov Eigenvalue Routines}
One of the simplest and easiest to code algorithms for finding eigenvalues is the power method. It is given by the following iteration \cite{MeyerLinAlg}
\begin{align}
u_{k+1}=\frac{Au_{k}}{|Au_{k}|}, \sigma_{k}=\frac{u_{k}^{t}Au_{k}}{u_{k}^{t}u_{k}}.
\end{align}
The reasoning for this algorithm is very simple. If we suppose that the operator $A$ has a spectrum of eigenvalues such that $\lambda_{1} > \lambda_{2} > \ldots $ then by the spectral theorem we can re-write the operator $A$ as
\begin{align}
A = \lambda_{1}\bm{u}_{1}\bm{u}_{1}^{T} + \lambda_{2}\bm{u}_{2}\bm{u}_{2}^{T} + \lambda_{3}\bm{u}_{3}\bm{u}_{3}^{T} + \ldots 
\end{align} 
where $\bm{u}_{i}$ is the eigenvector associated with $\lambda_{i}$. Now recall that by the spectral theorem we have the following \cite{MeyerLinAlg}
\begin{align}
\left(\frac{A}{\lambda_{1}}\right)^{n} = \bm{u}_{1}\bm{u}_{1}^{T} + \left(\frac{\lambda_{2}}{\lambda_{1}}\right)^{n}\bm{u}_{2}\bm{u}_{2}^{T} + \left(\frac{\lambda_{3}}{\lambda_{1}}\right)^{n}\bm{u}_{3}\bm{u}_{3}^{T} + \ldots 
\end{align}
and since there is an ordering on the eigenvalues we have that $\lambda_{i}/\lambda_{1}<1$ and hence as $n\rightarrow\infty$ these terms will vanish. Thus if we consider a random initial vector $\bm{b}_{0}$ we have that 
\begin{align}
\frac{A^{n}\bm{b}_{0}}{\lambda_{1}^{n}}\rightarrow \bm{u}_{1}\bm{u}_{1}^{T}\bm{b}_{0}, n\rightarrow\infty.
\end{align}
Now in practice we don't have $\lambda_{1}$ but if we replace $\lambda_{1}$ with the norm of $|A^{n}\bm{b}_{0}|$ we can still get a good approximation \cite{MeyerLinAlg}. By repeated iteration we obtain the scheme from above. This very simple algorithm is able to give us the leading eigenvalue and eigenvector starting from any random initial vector. A rigorous derivation is contained in any good numerical linear algebra book, e.g. Trefethen and Bau \cite{trefethen1997}.

One drawback of this method is that we are throwing away a lot of information at each iteration. It does not seem unreasonable that maybe these previous approximations can tell us something.Utilising this extra information is the key of Krylov methods. First, we define a Krylov sequence to be 
\begin{align}
\{u,Au,A^{2}u,\ldots,A^{n-1}u\}
\end{align}
This definition is just taking the first $n$ iterations of the power method and defining a sequence of them. Likewise we can also define the Krylov matrix as 
\begin{align}
K_{n}(A,u)=(u|Au|A^{2}u|\ldots|A^{n-1}u)
\end{align}
where we are defining a new matrix whose $i$th column is the $i$ iteration. This Krylov matrix allows us to compute the sub-dominant modes.  

Consider the following matrix, $L=K^{-1}AK$, which is the similarity transformation of $A$ by $K$. Recall that the eigenvalues of similar matrices are the same \cite{MeyerLinAlg}. This is the key idea of the Krylov methods, if we apply a similarity transform to our original matrix $A$ by $K$ we can instead compute the eigenvalues of $L$. However we still are stuck with the inverse matrix $K^{-1}$. To compute this inverse, we orthogonalise $K$ (by Gram-Schmidt) and thus $K^{-1}=K^{t}$. 

So far everything has been exact since the size of $K$ must be the same size as $A$. Now we make the approximation that we can replace $K$ with $K_{n}$. Now taking the first $n$ iterations, and $n$ being small in some sense, let us define a new matrix $P$ as above
\begin{align}
P=K_{n}^{t}AK_{n}.
\end{align}
Now $P$ is no longer the same size as $A$. Clearly the similarity properties no longer hold but if the eigenvalues of $P$ are computed, they, rather amazingly, turn out to be a very good approximation to $A$. This leads to a very simple idea for finding dominant eigenvalues for A: simply choose a small $n$ and then find the eigenvalues of $P$ which is only $n\times n$ and there exist many good algorithms for finding the eigenvalues of small $n$. 

This amazing fact has been fairly well understood for symmetric $A$ and leads to the Lanczos algorithm \cite{MeyerLinAlg}. The non-symmetric case is less well understood and leads to the Arnoldi iteration \cite{MeyerLinAlg}. Similar ideas of using Krylov sequences and matrices form the theoretical basis for conjugate gradient and GMRES algorithms for solving linear systems \cite{watkins2007matrix}.


\subsection{Results}
\begin{figure}
\begin{center}
\includegraphics[scale=0.7]{final_data.png}
\caption{Krylov methods with two different choices of $n=50$  (left), $100$ (right) for $Re=10000, N=256, F_{h}=0.1$. Star denotes the dominant eigenvalues from the energy method, circle is the leading approximation from the Krylov method, triangle and x are the second and third eigenvalues respectively.}
\label{krylov_tests}
\end{center} 
\end{figure}
Fig.~\ref{krylov_tests} uses the Krylov method for finding sub dominant eigenvalues for the linear stability of a Lamb-Chaplygin dipole in a stratified fluid for $Re=10000, F_{h}=0.1, N=256$. The reason $N=256$ was chosen was for memory issues in MATLAB. Some tests of various sizes of $N$ showed some robustness although a thorough convergence investigation was not completed and the results were only compared to those of the previous sections. To find the Krylov sequence, the output from the simulation of the previous section was saved at certain times.  Selecting $n$ samples, they were orthogonalised using a Gram-Schmidt procedure  and used to form the Krylov matrix $K$. To evaluate $AK$ the linear code was simulated as normal and reformatted into a giant vector. Then $P$ was constructed and the eigenvalues of the resulting matrix were found using MATLAB's \texttt{eigs} routine. 

The results of Fig.~\ref{krylov_tests} are somewhat promising. As can be seen, doubling $n$ does improve convergence, although there are still numerical anomalies, such as the random jump in growth rate at $k_{z}=60,65$ for $n=100$. As mentioned above, the theory of Krylov methods for non-symmetric matrices is still not well developed and few rigorous results on errors have been derived. The leading eigenvalue is reproduced accurately up to just about the short-wave peak before the Krylov results decay faster. The most promising result is at the local minimum between the two peaks. Here there is a very smooth curve of the growth rates of the second and third eigenmodes whose values are close to the values of the leading eigenmode. This result suggests that at this local minimum, a different eigenmode might be the dominate eigenmode for the short wave instability than that of the zigzag peak. Returning to the structure, Fig.~\ref{evolution} supports this viewpoint as the zigzag and short-wave peak velocity fields are different and the oscillatory region in between suggests a combination of the two which might be the two different modes interchanging dominance and resulting in a mix of the two. 

Finally, some attempts were made use MATLABs \texttt{eigs} routine and passing the operator $A$ explicitly, instead of evaluating from a time series, were unsuccessful despite many hours of debugging. 

\section{Dimensional Analysis}
Motivated by the scale analysis of those presented in the Chapter 2 review \cite{lilly1983,rileylelong2000,bc2001,brethouwer2007}, we present a scaling analysis for small vertical scales as considered in the above numerical simulations. We consider the Boussinesq equations

\begin{align}
\frac{\partial \textbf{u}_{h}'}{\partial t'} + \textbf{u}_{h}'\cdot\nabla'_{h}\textbf{u}_{h}'+u_{z}'\frac{\partial \textbf{u}_{h}'}{\partial z'} &= -\frac{1}{\rho_{0}}\nabla'_{h}p' + \nu \nabla'^{2}\textbf{u}_{h}\label{scaling_horz},\\
\frac{\partial u_{z}'}{\partial t'} + \textbf{u}_{h}'\cdot\nabla'_{h}u_{z}'+u_{z}'\frac{\partial u_{z}'}{\partial z'} &= -\frac{1}{\rho_{0}}\frac{\partial p'}{\partial z'} - \frac{\rho' g}{\rho_{0}} + \nu \nabla'^{2}u_{z},\label{scaling_vert}\\
\nabla'_{h}\cdot\textbf{u}_{h}' + \frac{\partial u_{z}'}{\partial z'} &=0,\label{scaling_cont}\\
\frac{\partial \rho'}{\partial t'} + \textbf{u}_{h}'\cdot\nabla_{h}'\rho' + u_{z}'\frac{\partial \rho'}{\partial z'} + \frac{\partial \rho}{\partial z'}u_{z}'&=D\nabla^{2}\rho ',\label{scaling_moment}
\end{align}
where the primed notation denotes the dimensional variables in this section only. 

Following \cite{bc2001} let $U,W$ be the characteristic velocities in the horizontal and vertical directions, $L_{h},L_{v}$ be the corresponding characteristic length scales, $P$ be the pressure, and $R$ be density perturbation scales, not to be confused with the dipole radius $R$ from above. We assume, differing from the analysis of \cite{lilly1983,bc2001}, that in addition to $U,L_{h}$ being imposed on the system, we also impose a separate vertical scale $L_{v}$. This scaling is motivated by the above numerical simulations where we impose a vertical length scale through the vertical wavenumber $k_{z}$. The aspect ratio $\delta=L_{v}/L_{h}$ is assumed to be small, i.e. $\delta<1$. We define the horizontal Froude number to be $F_{h}=U/NL_{h}$, which is also assumed to be small. Following the above numerical simulations, let $\delta < F_{h}$, which we can also write as $L_{v} < U/N$, i.e. vertical scales are assumed to be smaller than the buoyancy scale. We now define the advective time scale $T=L_{h}/U$. 
To determine the characteristic scale of $W$, we are left with two choices: imposing the scaling from the continuity equation or from the density equation. Previous work \cite{bc2001} chose the latter and obtained a characteristic velocity 
\begin{align}
W \lesssim \frac{RF_{h}g}{\rho_{0}N}\label{scaling1}. 
\end{align}
By contrast, we use the continuity equation (\ref{scaling_cont}), which implies
\begin{align}
W \lesssim \delta U\label{scaling2}.
\end{align}
This scaling for $w$ is consistent with the assumption that $\delta < F_{h}$. Using (\ref{scaling2}), the vertical momentum equation (\ref{scaling_vert}) gives a density scaling of $R\sim \rho_{0}U^{2}/(gL_{v})$. Plugging this result into (\ref{scaling1}), we obtain $W\sim UF_{h}^{2}/\delta$. Because $\delta < F_{h}$ we have $U\delta < UF_{h}^{2}/\delta$ so our assumptions are consistent. Setting $W\sim U\delta$ the horizontal momentum equation (\ref{scaling_horz}) gives $P\sim \rho_{0}U^{2}$. Combining this all, we obtain the following scaling for the Boussinesq equations with $L_{v} < U/N$:  

\begin{align}
\textbf{u}_{h}' = U\textbf{u}_{h},\qquad u_{z}'=U\delta u_{z},\qquad \rho' =\frac{U^{2}\rho_{0}}{gL_{v}}\rho,\qquad p'=\rho_{0}U^{2}p, \nonumber\\
\textbf{x}=L_{h}x,\qquad z'=L_{v}z,\qquad t' = \frac{L_{h}}{U}{t},\qquad Re=\frac{UL_{h}}{\nu},\qquad Sc = \frac{\nu}{D}
\end{align} 
which leads to  
\begin{align}
\frac{\partial \textbf{u}_{h}}{\partial t} + \textbf{u}_{h}\cdot\nabla_{h}\textbf{u}_{h}+u_{z}\frac{\partial \textbf{u}_{h}}{\partial z} &= -\nabla_{h}p + \frac{1}{Re}\nabla_{h}^{2}\textbf{u}_{h} +  \frac{1}{\delta^{2}Re}\frac{\partial^{2}\textbf{u}_{h}}{\partial z^{2}},\\
\delta^{2}\left(\frac{\partial u_{z}}{\partial t} + \textbf{u}_{h}\cdot\nabla_{h}u_{z}+u_{z}\frac{\partial u_{z}}{\partial z}\right) &= -\frac{\partial p}{\partial z} - \rho' + \frac{\delta^{2}}{Re}\nabla_{h}^{2}u_{z} + \frac{1}{Re}\frac{\partial^{2}u_{z}}{\partial z^{2}},\\
\nabla_{h}\cdot\textbf{u}_{h} + \frac{\partial u_{z}}{\partial z} &=0,\\
\frac{\partial \rho'}{\partial t} + \textbf{u}_{h}\cdot\nabla_{h}\rho' + u_{z}\frac{\partial \rho'}{\partial z} -\frac{\delta^{2}}{F_{h}^{2}}u_{z}&=\frac{1}{ReSc}\nabla_{h}^{2}\rho + \frac{1}{\delta^{2}ReSc}\frac{\partial^{2}\rho}{\partial z^{2}},
\end{align}
which holds when $\delta<F_{h}\ll 1$. This suggests that for very small vertical scales with $\delta \ll  F_{h}$ the effects of stratification should be negligible. At such small vertical scales, density variation due to stratification would be negligible and thus we would not expect stratification to play an important role in the overall evolution. Additionally, the presence of the factors of $\delta$ in the denominator of the vertical viscous terms suggests that the effects of viscosity become more dominant at very small vertical scales.  

As a result of this scaling analysis we expect that the nature of the instability at short vertical scales to become independent of $F_{h}$ for large $k_{z}$. To test this hypothesis Fig.~\ref{sigma_scaling} shows growth rate as a function of $k_{z}$ for four sets of simulation with $Re=10{,}000$: $F_{h}=0.2,0.1,0.05$ and a new unstratified case with $F_{h}=\infty$ (note that, unlike in Fig.~\ref{FixReVaryFh}, we are not scaling $k_{z}$ by $F_{h}$). The growth rate curves appear to be converging for large $k_{z}$ where $\delta \ll F_{h}$, which agrees with the conclusion of the above scaling analysis. These large $k_{z}$ are well into the viscous damping range and as discussed above, the effects of viscosity become stronger and we observe a sharper decrease in the growth rate.

For the short-wave instability examined above, $\delta/F_{h}=1/(k_{z}F_{h})$ ranges from $\approx 0.5$ down to $0.1$, which is $<1$ but not $\ll 1$. As a result, we do not necessarily expect the characteristics of this instability to be independent of $F_{h}$ for the parameters considered here. Indeed, our stability analysis shows that the (unscaled) wavenumber $k_{z}$ of the short-wave peak is weakly dependent on $F_{h}$, through the $F_{h}^{1/5}$ factor in (\ref{buoyscale}). However, by examining even larger $k_{z}F_{h}$ (i.e. even smaller $\delta/F_{h}$), this scale analysis suggests that the nature of the short-wave instability will eventually become independent of $F_{h}$.  

\begin{figure}
\begin{center}
\includegraphics[scale=0.7]{sigma_kz_scaling}
\caption{Growth rate $\sigma$ as a function $k_{z}$ at $Re=10{,}000$ with $F_{h}=\infty$ (green), $F_{h}=0.2$ (blue), $F_{h}=0.1$ (black), $F_{h}=0.05$ (red)}
\label{sigma_scaling}
\end{center}
\end{figure}




%-------------------------------------------------------------------------------
%
%   NONLINEAR THEORY 
%
%-------------------------------------------------------------------------------
%%%%%%%%%%%%%%%%%%%%%%%%%%%%%%%%%%%%%%%%%%%%%%%%%%%%%%%%%%%%%%%%%%%%%%%%%%%%%%%%%%%%%%%%%%%%%%%%%%%%%
% NONLINEAR THEORY 
%
% 
%%%%%%%%%%%%%%%%%%%%%%%%%%%%%%%%%%%%%%%%%%%%%%%%%%%%%%%%%%%%%%%%%%%%%%%%%%%%%%%%%%%%%%%%%%%%%%%%%%%%
\chapter{Nonlinear theory}

In this chapter we extended the results of the linear stability analysis to a nonlinear analysis. Through numerical simulations, we suggest that the saturation level of the short-wave aspect ratio depends upon the aspect ratio. 

%\begin{itemize}
%\item Discuss nonlinear numerical set-up
%\item Just short wave results
%\item Scaling analysis 
%\item A few complete nonlinear simulations that capture all the dynamics.
%\item Appendix containing discussion of calculating theoretical quantities after conclusion
%\end{itemize}

\section{Set-up}
In this section we discuss some numerical tests of the code to verify that the nonlinear results are correctly resolving the samllest scales. 
In order to do a full DNS, we need to resolve the Kolmogorov scales. The Kolomogorov scale is given by (cite)
\begin{align}
\eta = \left(\frac{\nu^{3}}{\epsilon}\right)^{3}
\end{align}
where $\epsilon$ is the energy given by
\begin{align}
\epsilon \sim \frac{U^{3}}{R} 
\end{align}
where $U,R$ are the characteristic velocity and length respectively. Rewriting the Kolmogorov scale $\eta$ in terms of the Reynolds number by multipying by the characteristic length $R$, which we take to be unity as has been suggested from the linear results, to obtain
\begin{align}
\eta = \frac{1}{\Re^{3/4}}
\end{align}
For our simulations, the code seperates the horizontal and vertical resolutions, which we denote by $\Delta x$ and $\Delta z$. We want to choose the number of grid points to be such that $\Delta x \approx \Delta z$. 

To illustrate, we consider a test case that was run to determine whether or not to use $L=9$ or $L=5$ for the box size. Both have been used in practice (cite). Consider a grid with $N\times N \times n$ points, where we have explicitly seperated out the horizontal and vertical directions. Because we need to resolve the Kolmogorov scales, we have restrictions on the total number of grid points. We consider the case of $Re=2000$ and $F_{h}=0.2$. The Reynolds number tells us that that the Kolmogorov scale is
\begin{align}
\eta \sim \frac{1}{Re^{3/4}} \approx 0.003343,\qquad \Delta x \sim \eta 
\end{align}
and thus we want a grid spacing that is approximately $0.003343$. Since it is desirable for $N$ to be a power of two, if $N=1024$ we have that 
\begin{align}
\dx = \frac{L}{1024}.
\end{align}
If $L=9$ then $\dx=0.00878$ and if $L=5$ then $\dx=0.00488$. It is clear that it is more desirable to pick $L=5$ because the grid spacing is closer to the $\eta$. 

Now we need to set the vertical resolution. It is important that we have the same resolution for the vertical and horizontal so $\dx \approx \Delta z$. Since we are interested in investigating the short-wave instability the vertical scale is 
\begin{align}
H \sim \frac{2\pi}{k_{z}}.
\end{align}
In this case, the greatest growth rate occurs at the wavenumber $k_{z}=20$ and thus the vertical scale is $H=0.314$. Now setting $\Delta z \approx \dx$ for both $L=5,9$ we find that for $L=9$ if $n=32$ then $\Delta z=0.00982$ which is close to $\dx=0.00878$. If $L=5$ then if $n=64$ then $\Delta z=0.00491$ which is close to $\dx=0.00488$. Thus to resolve the Kolmogorov scale as much as possible, we should choose $L=5$ with $N=1024, n=64$over $L=9$ with $N=1024, n=32$. However when the code is actually run $L=5$ takes roughly $30.8$ hours of real time versus $19.5$ hours of real time. Additionally, due to the way the code is set-up, $L=5$ requires twice as many processors as $L=9$. Additionally, due to the way the code is set-up, $L=5$ requires twice as many processors as $L=9$ which requires more resources. Thus a trade-off must be made between running at a higher resolution but taking longer versus running at a lower resolution, in the processing potentially not resolving the Kolmogorov scale, but running faster. 

\begin{figure}
\begin{center}
\includegraphics[width=\textwidth]{energy_test}
\caption{Time series of the potential energy (circle) and kinetic energy (star) for $L=5$ (blue) and $L=9$ (black).}
\label{test_energy}
\end{center}
\end{figure}
In Fig.~\ref{test_energy} we demonstrate two tests run that were done to determine what resolution to use. Qualitatively, both curves are very similar, with the $L=5$ curves shifted upwards by a constant. This is to be expected because by using a larger domain size, less of the domain is focused on the dipole and thus there is less energy in this dipole. When we decrease the domain size, the dipole takes up a larger percentage of the grid and thus has a larger energy. 

%\begin{figure}
%\begin{center}
%\includegraphics[width=\textwidth]{vert_spectrum_test}
%\caption{Time series of the growth rate, obtained from the derivative of the energy, for $F_{h}=0.1$ and $Re=10{,}000$. Panel (a) is $k_{z}=20$ and Panel (b) is $k_{z}60$.}
%\label{vert_spec_test}
%\end{center}
%\end{figure}
\section{Results}
The nonlinear evolution of the short-wave instability tells us how the zigzag instability and the short-wave instability interact. Investigations by Waite and Smolarkiewicz into the breakdown of the Lamb-Chaplygin dipole into turbulence and observed that the energy of the zigzag instability grew such that it became the same order of magntiude as the kinetic energy. In these simulations, no short-wave instabiltiies were observed despite having a similar growth rate. We present results that suggest that the short-wave instability saturates at a level proportional to the aspect ratio $\delta$.

To investigate the saturation level, nonlinear simulations were run with an initially small, $\epsilon=0.01$, random sinusodial perturbation of the initial density. Figure (blah) is an example of the evolution of the energy for a given set of parameters. As can be observed in this figure the saturation level occurs at roughly $T=$.  
\begin{align}
\text{saturation} = \frac{E_{3D}}{E_{2D}}
\end{align}
where we find the maximum $E_{3D}$ at later times and divide that time by the kinetic energy. 

\begin{figure}
\begin{center}
\includegraphics[width=\textwidth]{re2000_fh02_saturations} 
\caption{Saturation levels for a range of aspect ratios for $Re=2000$ and $F_{h}=0.2$.}
\label{re2000sat}
\end{center}
\end{figure}
\begin{figure}
\begin{center}
\includegraphics[width=\textwidth]{re5000_fh02_saturations} 
\caption{Saturation levels for a range of aspect ratios for $Re=5000$ and $F_{h}=0.2$. }
\label{re5000sat}
\end{center}
\end{figure}
Fig.~\ref{re2000sat} and Fig.~\ref{re5000sat} demonstrate the saturation level for $F_{h}=0.2$ and $Re=2000,5000$ respectively. The reference line is a slope of 2. In Fig.~\ref{re5000sat} there is some variation in the data points as there is no clear saturation level. Fig blah demonstrates an example of this. From these curves, it suggests that  
\begin{align}
\frac{E_{3Dsat}}{E_{2Dsat}} \sim \delta^{2}.
\end{align}
This scaling suggests that as the aspect ratio gets smaller, the saturation level decreases. 

The scaling suggested has a theoretical backing. In investigations by Ngan, et al \cite{ngan2005} into quasi-two dimensional turbulence, they determined that the saturation level of a perturbation depended linearly upon the aspect ratio. We review their derivation. Ngan et. al \cite{ngan2005} consider quasi-two dimensional turbulence, which they define to be the three-dimesionalisation of a two-dimensional flow. The study of such flows are motivated by geophysical applications were aspects ratios range from $\delta\sim 0.01-0.1$\cite{ngan2005}. 

Ngan et al. now consider a simple scaling analysis \cite{ngan2005}. Initially the time scales of the 2D base flow is long compared to that of the 3D base flow. In our results, we verify this. In figure blah note that the time scale of the 2D perturabtion, here the top line, is constant while the timescale of the 3D flow is much shorter, which is exemplified by the change that the 3D perturbation goes through over the first 20 or seconds. However once the 3D perturbation saturates, it stops growing and its timescale becomes similar to that of the 2D flowa.  Now consider the following timescales for these flows \cite{ngan2005} as
\begin{align}
T_{2D} = \frac{U}{L},\qquad T_{3D} =\frac{u}{H}
\end{align}
where $U$ is the characteristic velocity of the 2D flow and $u$ is the characteristic velocity. Thus at saturation $T_{2D}\sim T_{3D}$ and thus we have that
\begin{align}
\frac{U}{L}\sim\frac{u}{H} \Rightarrow \frac{u}{U} \sim \frac{H}{L} = \delta .
\end{align}
However we are considering the energy and thus we would have to square both sides to get the energy resulting in
\begin{align}
\frac{E_{2D}}{E_{3D}} \sim \delta^{2}
\end{align}
which is the result suggested by Figs.~\ref{re2000sat} and Fig.~\ref{re5000sat}. Thus, the short-wave instability, despite having a similar growth rate to that of the zigzag instability, is saturated at a level proportional to the aspect ratio of the flow, which in stratified flows, is small. 

%Furthermore, in the scaling analysis derived in the previous chapter, if $\delta < F_{h} \ll 1$ then the startification 


%To determine this linear relationship, they consider the following flow
%\begin{align}
%\bm{v}(x,y,z,t) = \bm{U}(x,y,t)+\bm{u}(x,y,z,t)
%\end{align}
%where $\bm{U}$ is a 2D base flow that satisfies the 2D Navier-Stokes equations and $\bm{u}$ is a 3D perturbation. 


%-------------------------------------------------------------------------------
%
%   CONCLUSIONS
%
%-------------------------------------------------------------------------------
%\chapter{Conclusions}


%\appendix
%
%% Add a title page before the appendices and a line in the Table of Contents
%\chapter*{APPENDICES}
%\addcontentsline{toc}{chapter}{APPENDICES}
%%======================================================================
%\chapter[PDF Plots From Matlab]{Matlab Code for Making a PDF Plot}
%\label{AppendixA}
%% Tip 4: Example of how to get a shorter chapter title for the Table of Contents 
%%======================================================================
%\section{Using the GUI}
%Properties of Matab plots can be adjusted from the plot window via a graphical interface. Under the Desktop menu in the Figure window, select the Property Editor. You may also want to check the Plot Browser and Figure Palette for more tools. To adjust properties of the axes, look under the Edit menu and select Axes Properties.
%
%To set the figure size and to save as PDF or other file formats, click the Export Setup button in the figure Property Editor.
%
%\section{From the Command Line} 
%All figure properties can also be manipulated from the command line. Here's an example: 
%\begin{verbatim}
%x=[0:0.1:pi];
%hold on % Plot multiple traces on one figure
%plot(x,sin(x))
%plot(x,cos(x),'--r')
%plot(x,tan(x),'.-g')
%title('Some Trig Functions Over 0 to \pi') % Note LaTeX markup!
%legend('{\it sin}(x)','{\it cos}(x)','{\it tan}(x)')
%hold off
%set(gca,'Ylim',[-3 3]) % Adjust Y limits of "current axes"
%set(gcf,'Units','inches') % Set figure size units of "current figure"
%set(gcf,'Position',[0,0,6,4]) % Set figure width (6 in.) and height (4 in.)
%cd n:\thesis\plots % Select where to save
%print -dpdf plot.pdf % Save as PDF
%\end{verbatim}

%----------------------------------------------------------------------
% END MATERIAL
%----------------------------------------------------------------------

% B I B L I O G R A P H Y
% -----------------------

% The following statement selects the style to use for references.  It controls the sort order of the entries in the bibliography and also the formatting for the in-text labels.
\bibliographystyle{plain}
% This specifies the location of the file containing the bibliographic information.  
% It assumes you're using BibTeX (if not, why not?).
\cleardoublepage % This is needed if the book class is used, to place the anchor in the correct page,
                 % because the bibliography will start on its own page.
                 % Use \clearpage instead if the document class uses the "oneside" argument
\phantomsection  % With hyperref package, enables hyperlinking from the table of contents to bibliography             
% The following statement causes the title "References" to be used for the bibliography section:
\renewcommand*{\bibname}{References}

% Add the References to the Table of Contents
\addcontentsline{toc}{chapter}{\textbf{References}}

\bibliography{thesis_bib}
% Tip 5: You can create multiple .bib files to organize your references. 
% Just list them all in the \bibliogaphy command, separated by commas (no spaces).

% The following statement causes the specified references to be added to the bibliography% even if they were not 
% cited in the text. The asterisk is a wildcard that causes all entries in the bibliographic database to be included (optional).
\nocite{*}

\end{document}

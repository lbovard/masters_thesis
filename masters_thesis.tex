\documentclass[letterpaper,12pt,titlepage,oneside,final]{book}
\newcommand{\package}[1]{\textbf{#1}} % package names in bold text
\newcommand{\cmmd}[1]{\textbackslash\texttt{#1}} % command name in tt font 
\newcommand{\href}[1]{#1} % does nothing, but defines the command so the
\usepackage{ifthen}
\newboolean{PrintVersion}
\setboolean{PrintVersion}{false} 
\usepackage{amsmath,amssymb,amstext,mathtools,amsfonts,amsthm} % Lots of math symbols and environments
\usepackage{bm}
\usepackage[pdftex]{graphicx} % For including graphics N.B. pdftex graphics driver 
\usepackage[pdftex,letterpaper=true,pagebackref=false]{hyperref} % with basic options
		% N.B. pagebackref=true provides links back from the References to the body text. This can cause trouble for printing.
\hypersetup{
    plainpages=false,       % needed if Roman numbers in frontpages
    pdfpagelabels=true,     % adds page number as label in Acrobat's page count
    bookmarks=true,         % show bookmarks bar?
    unicode=false,          % non-Latin characters in Acrobat’s bookmarks
    pdftoolbar=true,        % show Acrobat’s toolbar?
    pdfmenubar=true,        % show Acrobat’s menu?
    pdffitwindow=false,     % window fit to page when opened
    pdfstartview={FitH},    % fits the width of the page to the window
    pdftitle={uWaterloo\ LaTeX\ Thesis\ Template},    % title: CHANGE THIS TEXT!
%    pdfauthor={Author},    % author: CHANGE THIS TEXT! and uncomment this line
%    pdfsubject={Subject},  % subject: CHANGE THIS TEXT! and uncomment this line
%    pdfkeywords={keyword1} {key2} {key3}, % list of keywords, and uncomment this line if desired
    pdfnewwindow=true,      % links in new window
    colorlinks=true,        % false: boxed links; true: colored links
    linkcolor=blue,         % color of internal links
    citecolor=green,        % color of links to bibliography
    filecolor=magenta,      % color of file links
    urlcolor=cyan           % color of external links
}
\ifthenelse{\boolean{PrintVersion}}{   % for improved print quality, change some hyperref options
\hypersetup{	% override some previously defined hyperref options
%    colorlinks,%
    citecolor=black,%
    filecolor=black,%
    linkcolor=black,%
    urlcolor=black}
}{} % end of ifthenelse (no else)
\setlength{\marginparwidth}{0pt} % width of margin notes
\setlength{\marginparsep}{0pt} % width of space between body text and margin notes
\setlength{\evensidemargin}{0.125in} % Adds 1/8 in. to binding side of all 
\setlength{\oddsidemargin}{0.125in} % Adds 1/8 in. to the left of all pages
\setlength{\textwidth}{6.375in} % assuming US letter paper (8.5 in. x 11 in.) and 
\raggedbottom
\setlength{\parskip}{\medskipamount}
\renewcommand{\baselinestretch}{1} % this is the default line space setting
\let\origdoublepage\cleardoublepage
\newcommand{\clearemptydoublepage}{%
\clearpage{\pagestyle{empty}\origdoublepage}}
\let\cleardoublepage\clearemptydoublepage

%======================================================================
%   L O G I C A L    D O C U M E N T -- the content of your thesis
%======================================================================
\begin{document}

% For a large document, it is a good idea to divide your thesis
% into several files, each one containing one chapter.
% To illustrate this idea, the "front pages" (i.e., title page,
% declaration, borrowers' page, abstract, acknowledgements,
% dedication, table of contents, list of tables, list of figures,
% nomenclature) are contained within the file "uw-ethesis-frontpgs.tex" which is
% included into the document by the following statement.
%----------------------------------------------------------------------
% FRONT MATERIAL
%----------------------------------------------------------------------
%% T I T L E   P A G E
% -------------------
% Last updated May 24, 2011, by Stephen Carr, IST-Client Services
% The title page is counted as page `i' but we need to suppress the
% page number.  We also don't want any headers or footers.
\pagestyle{empty}
\pagenumbering{roman}

% The contents of the title page are specified in the "titlepage"
% environment.
\begin{titlepage}
        \begin{center}
        \vspace*{1.0cm}

        \Huge
        {\bf Short-Wave Vortex Instabilities in Stratified Flow}

        \vspace*{1.0cm}

        \normalsize
        by \\

        \vspace*{1.0cm}

        \Large
        Luke Bovard \\

        \vspace*{3.0cm}

        \normalsize
        A thesis \\
        presented to the University of Waterloo \\ 
        in fulfillment of the \\
        thesis requirement for the degree of \\
        Master of Mathematics\\
        in \\
        Applied Mathematics \\

        \vspace*{2.0cm}

        Waterloo, Ontario, Canada, 2013 \\

        \vspace*{1.0cm}

        \copyright\ Luke Bovard 2013 \\
        \end{center}
\end{titlepage}

% The rest of the front pages should contain no headers and be numbered using Roman numerals starting with `ii'
\pagestyle{plain}
\setcounter{page}{2}

\cleardoublepage % Ends the current page and causes all figures and tables that have so far appeared in the input to be printed.
% In a two-sided printing style, it also makes the next page a right-hand (odd-numbered) page, producing a blank page if necessary.
 


% D E C L A R A T I O N   P A G E
% -------------------------------
  % The following is the sample Delaration Page as provided by the GSO
  % December 13th, 2006.  It is designed for an electronic thesis.
  \noindent
I hereby declare that I am the sole author of this thesis. This is a true copy of the thesis, including any required final revisions, as accepted by my examiners.

  \bigskip
  
  \noindent
I understand that my thesis may be made electronically available to the public.

\cleardoublepage
%\newpage

% A B S T R A C T
% ---------------

\begin{center}\textbf{Abstract}\end{center}
Stratified flow is the essential underlying physical model for atmospheric and stratified flow. In recent years, the study of stratified turbulence has been more thoroughly investigated due to its difference from classical turbulence. As a first step to investigating the mechanisms of turbulence, linear stability plays a critical role in determining under what conditions a flow remains stable or becomes turbulent. In the study of transition to stratified turbulence, a common vortex model, known as the Lamb-Chaplygin dipole, is used to investigate the conditions under which stratified flow transitions to turbulence. Numerous investigations have determined that a critical length scale, known as the buoyancy length, plays a key role in the breakdown and transition to stratified turbulence. At this buoyancy length scale, an instability unique to stratified flow, the zigzag instability provides the mechanisms for this break down.  In this thesis we discover and investigate a new instability of the Lamb-Chaplyin dipole that exists at the sub-buoyancy scale. Through numerical linear stability analysis we show that this short-wave instability exhibits growth rates similar to that of the zigzag instability. We conclude with nonlinear studies of this short-wave instability and demonstrate this new instability saturates at a level proportional to the aspect ratio. 
\cleardoublepage
%\newpage

% A C K N O W L E D G E M E N T S
% -------------------------------

\begin{center}\textbf{Acknowledgements}\end{center}
Thanks to my supervisor Dr. Michael Waite. Good luck with the new addition to your family. Additional thanks goes to Dr. Marek Statsna and Dr. Francis Poulin. 

Obligatory thanks to everyone else who I have learnt and talked to over all these year. There are too many to name, so I won't. Just know that if you knew me I'd probably thank you. This saves me having to remember people and just cover everyone in one full swoop.

The work for this thesis was written under the music of Queen, Taylor Swift, The Beatles, Dire Straits, and Billy Joel. I kept statistics and it works out to be a few thousands plays of each, which is probably about 4 days straight of music per artists!

Richard you wanted me to add you punching Chris by mistake many years ago to this thesis. Here you go.

Special thanks to Patricia for making the last few months bearable. It made writing this thesis much easier. Thanks et merci y gracias por todo. 

Finally, and most importantly, I dedicate this thesis to the two best teachers I've ever had. Without both of them, who knows where I'd be now. 
\cleardoublepage
%\newpage

% D E D I C A T I O N
% -------------------

\begin{center}\textbf{Dedication}\end{center}
\begin{center}To Dr. Edward Vrscay and Dr. Jamal Sakhr.\end{center}
\cleardoublepage

%\newpage

% T A B L E   O F   C O N T E N T S
% ---------------------------------
\renewcommand\contentsname{Table of Contents}
\tableofcontents
\cleardoublepage
\phantomsection
%\newpage

% L I S T   O F   T A B L E S
% ---------------------------
\addcontentsline{toc}{chapter}{List of Tables}
\listoftables
\cleardoublepage
\phantomsection		% allows hyperref to link to the correct page
%\newpage

% L I S T   O F   F I G U R E S
% -----------------------------
\addcontentsline{toc}{chapter}{List of Figures}
\listoffigures
\cleardoublepage
\phantomsection		% allows hyperref to link to the correct page
%\newpage

% L I S T   O F   S Y M B O L S
% -----------------------------
% To include a Nomenclature section
% \addcontentsline{toc}{chapter}{\textbf{Nomenclature}}
% \renewcommand{\nomname}{Nomenclature}
% \printglossary
% \cleardoublepage
% \phantomsection % allows hyperref to link to the correct page
% \newpage

% Change page numbering back to Arabic numerals
\pagenumbering{arabic}



%----------------------------------------------------------------------
% MAIN BODY
%----------------------------------------------------------------------
% Because this is a short document, and to reduce the number of files
% needed for this template, the chapters are not separate
% documents as suggested above, but you get the idea. If they were
% separate documents, they would each start with the \chapter command, i.e, 
% do not contain \documentclass or \begin{document} and \end{document} commands.
%======================================================================
\chapter{Introduction}
\begin{itemize}
	\item Brief introduction to the topic
\end{itemize}
%======================================================================
THIS IS THE BEGINNING



%-------------------------------------------------------------------------------
%
%  NUMERICAL STUFF AND BOUSSINESSQ STUFF
%
%-------------------------------------------------------------------------------
\chapter{Numerical Stuff and Introductionary Theory}
\begin{itemize}
\item introduction to the boussinesq approximation 
\item In this section I should introduce the spectral method and the numerical scheme
\item Dealiasing tests 
\item  diffusion test
\end{itemize}
\section{Boussinesq approximation}

The equations can be written in the dimensional from as (add derivation, Kundu? Batchelor? Others?)

\begin{align}
\frac{\partial \bm{u}}{\partial t} + \bm{u}\cdot \nabla \bm{u} = -\frac{1}{\rho_{0}}\nabla p - \frac{\rho g}{\rho_{0}}\hat{\bm{e}}_{z} + \nu \nabla^{2}\bm{u} \label{boussinesq1}\\
\nabla \cdot \bm{u} =0 \label{boussinesq2}\\
\frac{\partial \rho}{\partial t} + \bm{u}\cdot \nabla \rho = \kappa \nabla^{2}\rho - \frac{\partial \bar{\rho}}{\partial z} w\label{boussinesq3}
\end{align}
where we have the following dimensional variables
\begin{itemize}
\item $\textbf{u}(x,y,z)=(u(x,y,z),v(x,y,z),w(x,y,z))$ is the velocity field in the $x,y,z$-directions directions respectively,
\item $p(x,y,z)$ is the pressure field,
\item $\rho(x,y,z)$ is a perturbation density,
\item $\rho_{0}$ is a constant reference density,
\item $\bar{\rho}$ is ... ? ,
\item $g$ is the gravitational constant,
\item $\nu$ is the constant kinematic velocity ,
\item $\kappa$ is the constant molecular diffusivity.
\end{itemize}
Equations (\ref{boussinesq1})-(\ref{boussinesq3}) are a set of five coupled partial differential equations for the unknowns $\textbf{u},p,\rho$. Herein, we will refer to equation (\ref{boussinesq1}) as the velocity equations, equation (\ref{boussinesq2}) as the continuity equation, and equation (\ref{boussinesq3}) as the density equation. 

Maybe go onto non-dimensionalisation? 

We take the above equations (Ref) as the starting point for our simulations. 


%-------------------------------------------------------------------------------
%
%   LINEAR THEORY 
%
%-------------------------------------------------------------------------------
\chapter{Linear stuff} 
\begin{itemize}
	\item derivation of the linear equations
	\item 
\end{itemize} 

\chapter{Nonlinear Stuff}


\chapter{Conclusions}


\appendix

% Add a title page before the appendices and a line in the Table of Contents
\chapter*{APPENDICES}
\addcontentsline{toc}{chapter}{APPENDICES}
%======================================================================
\chapter[PDF Plots From Matlab]{Matlab Code for Making a PDF Plot}
\label{AppendixA}
% Tip 4: Example of how to get a shorter chapter title for the Table of Contents 
%======================================================================
\section{Using the GUI}
Properties of Matab plots can be adjusted from the plot window via a graphical interface. Under the Desktop menu in the Figure window, select the Property Editor. You may also want to check the Plot Browser and Figure Palette for more tools. To adjust properties of the axes, look under the Edit menu and select Axes Properties.

To set the figure size and to save as PDF or other file formats, click the Export Setup button in the figure Property Editor.

\section{From the Command Line} 
All figure properties can also be manipulated from the command line. Here's an example: 
\begin{verbatim}
x=[0:0.1:pi];
hold on % Plot multiple traces on one figure
plot(x,sin(x))
plot(x,cos(x),'--r')
plot(x,tan(x),'.-g')
title('Some Trig Functions Over 0 to \pi') % Note LaTeX markup!
legend('{\it sin}(x)','{\it cos}(x)','{\it tan}(x)')
hold off
set(gca,'Ylim',[-3 3]) % Adjust Y limits of "current axes"
set(gcf,'Units','inches') % Set figure size units of "current figure"
set(gcf,'Position',[0,0,6,4]) % Set figure width (6 in.) and height (4 in.)
cd n:\thesis\plots % Select where to save
print -dpdf plot.pdf % Save as PDF
\end{verbatim}

%----------------------------------------------------------------------
% END MATERIAL
%----------------------------------------------------------------------

% B I B L I O G R A P H Y
% -----------------------

% The following statement selects the style to use for references.  It controls the sort order of the entries in the bibliography and also the formatting for the in-text labels.
\bibliographystyle{plain}
% This specifies the location of the file containing the bibliographic information.  
% It assumes you're using BibTeX (if not, why not?).
\cleardoublepage % This is needed if the book class is used, to place the anchor in the correct page,
                 % because the bibliography will start on its own page.
                 % Use \clearpage instead if the document class uses the "oneside" argument
\phantomsection  % With hyperref package, enables hyperlinking from the table of contents to bibliography             
% The following statement causes the title "References" to be used for the bibliography section:
\renewcommand*{\bibname}{References}

% Add the References to the Table of Contents
\addcontentsline{toc}{chapter}{\textbf{References}}

\bibliography{uw-ethesis}
% Tip 5: You can create multiple .bib files to organize your references. 
% Just list them all in the \bibliogaphy command, separated by commas (no spaces).

% The following statement causes the specified references to be added to the bibliography% even if they were not 
% cited in the text. The asterisk is a wildcard that causes all entries in the bibliographic database to be included (optional).
\nocite{*}

\end{document}

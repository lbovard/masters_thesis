\chapter{Numerical Stuff and Introductionary Theory}
\begin{itemize}
\item introduction to the boussinesq approximation 
\item In this section I should introduce the spectral method and the numerical scheme
\item Dealiasing tests 
\item  diffusion test
\item hyperviscosity
\end{itemize}
\section{Boussinesq approximation}

The equations can be written in the dimensional from as (add derivation, Kundu? Batchelor? Others?)

\begin{align}
\frac{\partial \bm{u}}{\partial t} + \bm{u}\cdot \nabla \bm{u} = -\frac{1}{\rho_{0}}\nabla p - \frac{\rho g}{\rho_{0}}\hat{\bm{e}}_{z} + \nu \nabla^{2}\bm{u} \label{boussinesq1}\\
\nabla \cdot \bm{u} =0 \label{boussinesq2}\\
\frac{\partial \rho}{\partial t} + \bm{u}\cdot \nabla \rho = \kappa \nabla^{2}\rho - \frac{\partial \bar{\rho}}{\partial z} w\label{boussinesq3}
\end{align}
where we have the following dimensional variables
\begin{itemize}
\item $\textbf{u}(x,y,z)=(u(x,y,z),v(x,y,z),w(x,y,z))$ is the velocity field in the $x,y,z$-directions directions respectively,
\item $p(x,y,z)$ is the pressure field,
\item $\rho(x,y,z)$ is a perturbation density,
\item $\rho_{0}$ is a constant reference density,
\item $\bar{\rho}$ is ... ? ,
\item $g$ is the gravitational constant,
\item $\nu$ is the constant kinematic velocity ,
\item $\kappa$ is the constant molecular diffusivity.
\end{itemize}
Equations (\ref{boussinesq1})-(\ref{boussinesq3}) are a set of five coupled partial differential equations for the unknowns $\textbf{u},p,\rho$. Herein, we will refer to equation (\ref{boussinesq1}) as the velocity equations, equation (\ref{boussinesq2}) as the continuity equation, and equation (\ref{boussinesq3}) as the density equation. 

Maybe go onto non-dimensionalisation? 

There is a difference between the linear and nonlinear equations (MIke's code uses a slightly modified version of Boussinesq that uses temperature) 
We take the above equations (Ref) as the starting point for our simulations. 

\section{Spectral Methods}

\section{Dealiasing} 
An important consideration in transforming between physical and Fourier space is the problem of aliasing.

Explain what it is! Blah blah wavenumbers meet and cross and cause problems.

Add derivation from Durran on this.

In the code we use a 2/3s rule however others have used other rules. To determine 

\section{1D Example} 
In this section we demonstrate the differences between solving PDEs in real space vs Fourier space. Consider the following one dimensional wave equation\cite{trefethen_spectral}
\begin{align}
\frac{\partial u}{\partial t} + c(x)\frac{\partial u}{\partial x} = 0,\qquad c(x)=\frac{1}{5}+\sin^{2}(x-1), \qquad u(x,0)=e^{-100(x-1)^{2}}, x\in[0,2\pi], t>0
\end{align}
where we are solving on a periodic domain. The physical interpretation of this equation is the simple one dimensional advection of a velocity field $u(x,t)$ by the fixed field $c(x)$. To illustrate issues relating to the spectral method, we will solve this problem in two ways: first we solve in the physical domain to demonstrate the idea of spectral differentiation and second to solve it in Fourier space to demonstrate the issues of dealiasing.

For a time-stepping scheme, we use a second-order Adams-Bashforth\cite{durran} scheme. Re-writing wave-equation as
\begin{align}
\frac{\partial u}{\partial t} = -c(x)\frac{\partial u}{\partial x} = F(u)
\end{align}
the Adams-Bashforth scheme is
\begin{align}
u^{n+1} = u^{n} + \frac{\dt}{2}[3F(u^{n})-F(u^{n-1})]
\end{align}
Since the Adams-Bashforth scheme uses a previous time-step, we will use forward Euler for the first time-step. To evalute $F(u)$ we will use the spectral differentiation, as discussed above. 
Figures go here.

Now we consider solving the equation in Fourier space. Taking the Fourier transform we obtain
\begin{align}
\frac{\partial \hat{u}}{\partial t} + \widehat{c(x)\frac{\partial u}{\partial x}}=0
\end{align}


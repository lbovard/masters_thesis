\chapter{Introduction}



Vortices play a fundamental role in the transition to turbulence by providing \francis{a mechanism to cascade energy} from large to small scales. In the atmosphere and ocean, vortices are strongly influenced by density stratification and the rotation of the earth. However, stratification dominates at intermediate length scales -- the small-scale end of the atmospheric mesoscale and the oceanic submesoscale --  which are small enough for the Coriolis effects to be weak, but large enough for the stable density stratification to be strong (e.g. \cite{riley2003,rileylindborg2013,waitebartello2004}).  There has recently been much work, using full direct numerical simulations of the Boussinesq equations with various initial configurations, to uncover the emergence and evolution of stratified turbulence from vortices \cite{augierbillant2011,augier2012,delonclebc2008,waitesmol2008}. Turbulence in this regime is governed by the Reynolds number $Re=UR/\nu$ as well as the horizontal Froude number $F_{h}=U/NR$, where $U$ is the characteristic velocity, $R$ is the characteristic horizontal length, $N$ is the Brunt-V\"{a}is\"{a}l\"{a} frequency, and $\nu$ is the kinematic viscosity. Because of this extra dependence on the Froude number, the underlying dynamics are not as well understood and a full picture of stratified turbulence is not complete \cite{lindborg2006,riley2003,rileylelong2000,rileylindborg2013,waitebartello2004}.  

In large-scale atmosphere and ocean simulations, it is difficult or impossible to resolve all possible processes. As a result, obtaining a proper parameterisation of small-scale phenomena is critical to correctly modelling the evolution. A useful approach to investigating these small-scale dynamics is to consider the transition problem in an idealised flow, which can elucidate the key features that govern the more comprehensive turbulence problem. One model that may be used to study the transition to stratified turbulence is that of a columnar counter-rotating vortex dipole. There is a large body of literature on the instability of vortex dipoles in unstratified fluids, including the Crow instability at large length scales (e.g. \cite{crow1970,widnall1974,leweke1998b}) and the elliptic instability at smaller scales (e.g. \cite{widnall1974,pierrehumbert1986,baily1986,waleffe1990}). In stratified fluids, laboratory and numerical experiments \francis{of dipoles have uncovered an instability unique to stratified flow}, the zigzag instability, so named due to the zigzag-like structure exhibited by the flow \cite{bc2000a,bc2000c}. The zigzag instability has a dominant vertical wavelength of around $U/N$, which is known as the buoyancy scale \cite{waite2011}. This instability has also been found in other flow configurations including co-rotating vortices \cite{otheguybc} and vortex arrays \cite{delonclebc2011}. The breakdown of this dipole into turbulence due to the growth and saturation of the zigzag instability has also been investigated \cite{waitesmol2008,augierbillant2011,delonclebc2008}. However, these studies mainly consider dipoles perturbed at the zigzag scale $U/N$, and do not investigate the growth of smaller vertical scale perturbations. Growth in such small-scale perturbations has been reported in nonlinear simulations \cite{waitesmol2008}. In this work we investigate the linear stability of the dipole at these small vertical scales. 

The buoyancy scale is an important length scale in stratified turbulence. It is the vertical scale at which the vertical Froude number is $\mathcal{O}(1)$ \cite{bc2001}, and it naturally emerges as the thickness of layers in stratified turbulence \cite{bc2001,waitebartello2004}. There is a direct transfer of energy, believed to be due to Kelvin-Helmholtz instability \cite{waite2011,augier2012}, from large horizontal scales into the buoyancy scale in stratified turbulence \cite{waite2011} and in the breakdown of the zigzag instability \cite{augier2012}. This breakdown generates small-scale turbulence which ultimately fills the spectrum at scales below the buoyancy scale. But it is possible that primary instabilities of the large-scale vortex may also directly excite vertical scales below the buoyancy scale. We investigate this possibility here. 

In this thesis we extend the linear stability analysis of Billant and Chomaz \cite{bc2000c} by investigating short, sub-buoyancy scale vertical wavelength perturbations of the Lamb-Chaplygin dipole in a stratified flow. In addition, we further investigate the nonlinear evolution of these sub-buoyancy scales. 

The thesis is organised as follows. Chapter 2 and 3 deal with the relevant background on fluid dynamics and numerical methods used. Chapter 4 investigates the linear stability of sub-buoyancy scales. Chapter 5 investigates the nonlinear evolution of the short-wave instability derived through linear theory. Parts of this thesis, including the introduction, linear stability background and results, and conclusions are based on the manuscript Bovard and Waite \cite{bovard2013}.

%The Lamb-Chaplygin dipole, an exact 2D solution to the Euler equations, is a good approximation to columnar counter-rotating dipole generated in lab experiments \cite{bc2000a}.  The work is presented as follows: in section 2 we present the numerical scheme and methodology, in section 3 we discuss the results of the numerical simulations and investigate some properties of the small-scale instability. Conclusions are discussed in the last section. 


%The buoyancy scale is $L_{b}=U/N$ although sometimes a factor of $2\pi$ is included as well. What does this number represent? In the atmosphere, $U\sim 10 \text{ m/s}$ and $N\sim 10^{-2} \text{ s}^{-1}$. 

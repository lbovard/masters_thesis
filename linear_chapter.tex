\chapter{Linear Theory}

\section{Introduction and Motivation}
The stability of flows has a long and rich history within fluid dynamics. The classic experiment into the stability of fluid flow is that of  Reynolds \cite{reynolds1883}. In this experiment, Reynolds injected dye into the laminar flow through a pipe. By varying the velocity of the flow through the pipe. If the velocity was sufficiently low, Reynolds was able to observe ``a beautiful straight line through the tube". At slightly higher velocities, the straight line behaviour remained near the initial part of the pipe but further down ``the streak would shift about the tube, but there was no appearance of sinuosity". By increasing the velocity significantly, the dye would again initially remain straight near the initial part of the tube, but instead of shifting about the tube "the colour band would all at once mix up with the surrounding water, and fill the rest of the tube with a mass of coloured water". 

What Reynolds observed was the transition and breakdown of a flow into turbulence. Through careful observation, he determined that the dimensionless quantity that governed the behaviour of the flow was the Reynolds number 
\begin{align}
Re =\frac{UL}{\nu},
\end{align}
a number which, as discussed above represents the ratio between the inertial and viscous terms in the Navier-Stokes equations. Reynolds found that if $Re<13000$ then the flow would remain stable. 

Derive Orr-Sommerfeld and use Plane pouseille flow to illustrate the ideas of linear stability 
An important question in any fluid flow is the question of whether or not the fluid flow is stable or unstable. 


To investigate short wave vortex instabilities in stratified fluids, we investigate the linear stability of a simple model (cite) used in stratified flow. Exploration of the linear stability flows can give insight into the transition to turbulence. 

\section{A Brief History of Vortex Instabilties}
The origin of vortex instabilities begins with Lord Kelvin who in 1880 studied perturbations to columnar vortices and determined that they were stable. For the next hundred or so years, the theory of vortex instability was quiet until the field was re-ignited by the investigations Crow\cite{crow1970}. Motivated by engineering applications, Crow was an aeronautical engineer, he discovered that for long wavelength perturbations of a pair of counter-rotating vortices that there was a symmetric deformation of the vortices. This worked extended a few years later by Widnall et. al \cite{widnall1974} to small wavelength perturbations. Further investigations into small wavelength perturbations were carried out by numerous others\cite{moore1975,tsai1976} (add in ref 3-7 bc1999) however these studies only considerd vortex filaments (define). It was noticed in these studies that the streamlines of the vortex became elliptical. Following through, Pierrehumbert\cite{pierrehumbert1986} investigated the simple case of a single 2D vortex subject to a constant 3D pertubation. Further work was done by Bayly \cite{bayly1986} and Waleffe\cite{waleffe1990}.

Something about the work of the japanese guys + experimental results. 

A complete and detailed history of the elliptical instability, including derivations and results of the above papers and nonlinear investigations, is presented in the review by Kerswell\cite{kerswell2002}.
\section{Zigzag Instability}
Motivated by the work above (cite), Billant and Chomaz investigate experimentally\cite{bc2000a}, theoretically\cite{bc2000b}, and numerically \cite{bc2000c} the evolution of a apair of columnar vortices in a stratified tank.

To run the experiment, Billant and Chomaz created a pair of long columnar vortices in a salt-stratified tank of water. To create the vortices, a pair of motor-controlled flaps were utilised. When the flaps were closed, a pair of counter-rotating vortices is produced. By fixing the initial angle of the flaps, $14\deg$, and varying the closing time of the flaps, the velocity of the pair of vortices could be changed. Interestingly, the decay of the vortices, roughly $90$s, was independent of the closing time of the flaps. Through the variation of the closing time, Billant and Chomaz were able to obtain different values for the Froude and Reynolds numbers, albeit not independently. 
\section{Formulation of Problem}
We consider the non-dimensional Boussinesq approximation to the Navier-Stokes equations in Cartesian co-ordinates (ref from intro)
\begin{align}
\frac{D\bm{u}}{Dt} = -\nabla p - \rho'\hat{\bm{e}}_{z} + \frac{1}{Re}\nabla^{2} \bm{u},\\
\nabla \cdot \bm{u}=0,\\
\frac{D\rho'}{Dt} -\frac{w}{F_{h}^{2}} = \frac{1}{ReSc}\nabla^{2} \rho',
\end{align}
where $D/Dt=\partial/\partial t + \bm{u}\cdot \nabla, \bm{u}=(u,v,w)$ is the velocity, $p$ is the pressure, and $\rho'$ is the density perturbation. We have non-dimensionalised by the characteristic velocity $U$, length $R$, time-scale $R/U$, pressure $\rho_{0}U^{2}$, density $\rho_{0}U^{2}/gR$, and defined $Sc=\nu /D$ as the Schmidt number, where $D$ is the mass diffusivity, $\rho_{0}$ is the background density, and $g$ is the gravitational constant. The Reynolds and horizontal Froude number are as defined above. The buoyancy frequency $N$, and hence the Froude number $F_{h}$, is assumed to be constant. 

\subsection{The Basic State}

As the basic state for linear stability analysis we use the Lamb-Chaplygin dipole in a comoving frame \cite{meleshko1994}. This dipole is a solution to the 2D inviscid Euler equations. This basic state is motivated by laboratory experiments \cite{bc2000a,leweke1998} which demonstrated that a vertically oriented Lamb-Chaplygin dipole is a good approximation to the vortex generated by two flaps closing in a tank of salt-stratified water. The dipole, in cylindrical coordinates, is given by the stream function
\begin{align}
\psi_{0}(r,\theta) = 
\begin{dcases}
-\frac{2}{\mu_{1}J_{0}(\mu_{1})}J_{1}(\mu_{1}r)\sin\theta & r\le 1,\\
-r\left(1-\frac{1}{r^{2}}\right)\sin\theta & r>1 ,
\end{dcases}
\end{align}
and the corresponding vertical vorticity $\omega_{z0}=\nabla_{h}^{2}\psi_{0}$
\begin{align}
\omega_{z0}(r,\theta) = \
\begin{dcases}
\mu_{1}^{2}\psi_{0}(r,\theta) & r\le 1,\\
0 & r>1,
\end{dcases}
\end{align}
where $J_{0},J_{1}$ are the zero and first order Bessel functions, $\mu_{1}\approx 3.38317$ is the first root of $J_{1}$, and $\nabla_{h}$ is the horizontal Laplacian. The basic state velocity is purely horizontal and is given by $\bm{u}_{h0}=\nabla_{h}\psi_{0}\times\hat{\bm{e}}_{z}$.

\subsection{Numerical Scheme}
We now write the fields as a basic state plus perturbations, denoted by $\sim$. Ignoring the viscous diffusion of the basic state \cite{drazinreid} and neglecting products of the perturbations, we obtain the following set of linear equations for the perturbations
\begin{align}
\frac{\partial \tilde{\bm{u}}}{\partial t} + \omega_{z0}\hat{\bm{e}}_{z}\times \tilde{\bm{u}}+\tilde{\boldsymbol{\omega}}\times \bm{u}_{h0} = -\nabla(\tilde{p}+\bm{u}_{h0} \cdot \tilde{\bm{u}}) - \tilde{\rho}'\hat{\bm{e}}_{z} + \frac{1}{Re}\nabla^{2}\tilde{\bm{u}},\label{nsl1}\\
\nabla\cdot\tilde{\bm{u}}=0,\\
\frac{\partial \tilde{\rho}'}{\partial t} + \bm{u}_{h0}\cdot \nabla_{h}\tilde{\rho}'-\frac{1}{F_{h}^{2}}\tilde{w} = \frac{1}{ScRe}\nabla^{2}\tilde{\rho}',\label{nsl3}
\end{align}
where $\tilde{\boldsymbol{\omega}}=\nabla \times \tilde{\bm{u}}$.

As stated above, the Lamb-Chaplygin dipole is oriented vertically. As a result we can separate the perturbation into the vertical and horizontal directions as 
\begin{align} 
[\tilde{\bm{u}},\tilde{p},\tilde{\rho}'](x,y,z,t) = [\bm{u},p,\rho'](x,y,t)e^{ik_{z}z} + \text{c.c.},
\end{align}
where c.c. is the complex conjugate. From here we can now take the 2D Fourier transform and define a projection operator $\textbf{P}(\textbf{k})$, with components $P_{ij}(\textbf{k})=\delta_{ij} - k_{i}k_{j}/k^{2}$ to eliminate pressure (e.g. Lesieur \cite{lesieur}) to obtain a set of equations for the Fourier coefficients 

\begin{align}
\frac{\partial \hat{\bm{u}}}{\partial t} = \textbf{P}(\textbf{k})[\widehat{\bm{u}\times \omega_{z0}\hat{\bm{e}}_{z}} + \widehat{\bm{u}_{h0}\times\bm{\omega}}-\hat{\rho}'\hat{\bm{e}}_{z}] - \frac{k^{2}}{Re}\hat{\bm{u}},\label{solve1}\\
\frac{\partial\hat{\rho}'}{\partial t} = -i\bm{k}_{h}\cdot\widehat{\bm{u}_{h0}\rho'} + \frac{1}{F_{h}^{2}}\hat{w}- \frac{k^{2}}{ScRe}\hat{\rho}',\label{solve2}
\end{align}
where $k_{z},Re,Sc,F_{h}$ are input parameters, $\bm{k}_{h}=(k_{x},k_{y})$ is the horizontal wavenumber and $k^{2}=k_{x}^{2}+k_{y}^{2}+k_{z}^{2}$ is the total wavenumber. 
\begin{itemize}
	\item derivation of the linear equations
	\item subdominant modes
\end{itemize} 

\section{Results}

\subsection{Growth Rates}
\subsection{Structure}
\subsection{Subdominant modes?} 
The above analysis only provides the leading eigenvalue. 
\section{Dimensional Analysis}

\section{Conclusion}


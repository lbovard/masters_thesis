\chapter{Linear Theory}

\section{Introduction}
An important question in any fluid flow is the question of whether or not the fluid flow is stable or unstable. 

throw in example from Trefethen? 

To investigate short wave vortex instabilities in stratified fluids, we investigate the linear stability of a simple model (cite) used in stratified flow. Exploration of the linear stability flows can give insight into the transition to turbulence. 

\section{Formulation}
We consider the non-dimensional Boussinesq approximation to the Navier-Stokes equations in Cartesian co-ordinates (ref from intro)
\begin{align}
\frac{D\bm{u}}{Dt} = -\nabla p - \rho'\hat{\bm{e}}_{z} + \frac{1}{Re}\nabla^{2} \bm{u},\\
\nabla \cdot \bm{u}=0,\\
\frac{D\rho'}{Dt} -\frac{w}{F_{h}^{2}} = \frac{1}{ReSc}\nabla^{2} \rho',
\end{align}
where $D/Dt=\partial/\partial t + \bm{u}\cdot \nabla, \bm{u}=(u,v,w)$ is the velocity, $p$ is the pressure, and $\rho'$ is the density perturbation. We have non-dimensionalised by the characteristic velocity $U$, length $R$, time-scale $R/U$, pressure $\rho_{0}U^{2}$, density $\rho_{0}U^{2}/gR$, and defined $Sc=\nu /D$ as the Schmidt number, where $D$ is the mass diffusivity, $\rho_{0}$ is the background density, and $g$ is the gravitational constant. The Reynolds and horizontal Froude number are as defined above. The buoyancy frequency $N$, and hence the Froude number $F_{h}$, is assumed to be constant. 

\section{The Basic State}
As the basic state for linear stability analysis we use the Lamb-Chaplygin dipole in a comoving frame \cite{meleshko1994}. This dipole is a solution to the 2D inviscid Euler equations. This basic state is motivated by laboratory experiments \cite{bc2000a,leweke1998} which demonstrated that a vertically oriented Lamb-Chaplygin dipole is a good approximation to the vortex generated by two flaps closing in a tank of salt-stratified water. The dipole, in cylindrical coordinates, is given by the stream function
\begin{align}
\psi_{0}(r,\theta) = 
\begin{dcases}
-\frac{2}{\mu_{1}J_{0}(\mu_{1})}J_{1}(\mu_{1}r)\sin\theta & r\le 1,\\
-r\left(1-\frac{1}{r^{2}}\right)\sin\theta & r>1 ,
\end{dcases}
\end{align}
and the corresponding vertical vorticity $\omega_{z0}=\nabla_{h}^{2}\psi_{0}$
\begin{align}
\omega_{z0}(r,\theta) = \
\begin{dcases}
\mu_{1}^{2}\psi_{0}(r,\theta) & r\le 1,\\
0 & r>1,
\end{dcases}
\end{align}
where $J_{0},J_{1}$ are the zero and first order Bessel functions, $\mu_{1}\approx 3.38317$ is the first root of $J_{1}$, and $\nabla_{h}$ is the horizontal Laplacian. The basic state velocity is purely horizontal and is given by $\bm{u}_{h0}=\nabla_{h}\psi_{0}\times\hat{\bm{e}}_{z}$.

We now write the fields as a basic state plus perturbations, denoted by $\sim$. Ignoring the viscous diffusion of the basic state \cite{drazinreid} and neglecting products of the perturbations, we obtain the following set of linear equations for the perturbations
\begin{align}
\frac{\partial \tilde{\bm{u}}}{\partial t} + \omega_{z0}\hat{\bm{e}}_{z}\times \tilde{\bm{u}}+\tilde{\boldsymbol{\omega}}\times \bm{u}_{h0} = -\nabla(\tilde{p}+\bm{u}_{h0} \cdot \tilde{\bm{u}}) - \tilde{\rho}'\hat{\bm{e}}_{z} + \frac{1}{Re}\nabla^{2}\tilde{\bm{u}},\label{nsl1}\\
\nabla\cdot\tilde{\bm{u}}=0,\\
\frac{\partial \tilde{\rho}'}{\partial t} + \bm{u}_{h0}\cdot \nabla_{h}\tilde{\rho}'-\frac{1}{F_{h}^{2}}\tilde{w} = \frac{1}{ScRe}\nabla^{2}\tilde{\rho}',\label{nsl3}
\end{align}
where $\tilde{\boldsymbol{\omega}}=\nabla \times \tilde{\bm{u}}$.

As stated above, the Lamb-Chaplygin dipole is oriented vertically. As a result we can separate the perturbation into the vertical and horizontal directions as 
\begin{align} 
[\tilde{\bm{u}},\tilde{p},\tilde{\rho}'](x,y,z,t) = [\bm{u},p,\rho'](x,y,t)e^{ik_{z}z} + \text{c.c.},
\end{align}
where c.c. is the complex conjugate. From here we can now take the 2D Fourier transform and define a projection operator $\textbf{P}(\textbf{k})$, with components $P_{ij}(\textbf{k})=\delta_{ij} - k_{i}k_{j}/k^{2}$ to eliminate pressure (e.g. Lesieur \cite{lesieur2008turbulence}) to obtain a set of equations for the Fourier coefficients 

\begin{align}
\frac{\partial \hat{\bm{u}}}{\partial t} = \textbf{P}(\textbf{k})[\widehat{\bm{u}\times \omega_{z0}\hat{\bm{e}}_{z}} + \widehat{\bm{u}_{h0}\times\bm{\omega}}-\hat{\rho}'\hat{\bm{e}}_{z}] - \frac{k^{2}}{Re}\hat{\bm{u}},\label{solve1}\\
\frac{\partial\hat{\rho}'}{\partial t} = -i\bm{k}_{h}\cdot\widehat{\bm{u}_{h0}\rho'} + \frac{1}{F_{h}^{2}}\hat{w}- \frac{k^{2}}{ScRe}\hat{\rho}',\label{solve2}
\end{align}
where $k_{z},Re,Sc,F_{h}$ are input parameters, $\bm{k}_{h}=(k_{x},k_{y})$ is the horizontal wavenumber and $k^{2}=k_{x}^{2}+k_{y}^{2}+k_{z}^{2}$ is the total wavenumber. 
\begin{itemize}
	\item derivation of the linear equations
	\item subdominant modes
\end{itemize} 


\section{Sub-Dominant modes}

The above analysis only provides the leading eigenvalue. 

\chapter{Conclusions}

In this thesis, we have investigated the linear stability of the Lamb-Chaplygin dipole for perturbations with small vertical scales. In particular, we have considered vertical scales from around the buoyancy scale $U/N$, where the zigzag instability occurs \cite{bc2000c,bc2001,waitebartello2004,waite2011}, down to the dissipation scale. We have discovered a short-wave instability that emerges at scales much smaller than the buoyancy scale. This instability can exhibit a growth rate that is comparable to, and possibly even greater than, that of the zigzag instability. Despite having a similar growth rate in some cases, the structure of the instability is qualitatively different that of the zigzag peak suggesting a different mechanism is governing the evolution.  We have discovered that the location of the peak depends upon a combination of the Reynolds and Froude numbers, specifically the buoyancy Reynolds number $Re_{b}$ which plays an important role in stratified fluids. The wavenumber of maximum growth rate for the short-wave instability is found to scale like $F_{h}k_{z}\sim Re_{b}^{2/5}$ for the range of $Re_{b}$ considered here. We expect this may change at even larger $Re_{b}$. By contrast, the maximum growth rate of the zigzag instability occurs for $F_{h}k_{z}\sim 1$ \cite{bc2000b}. As a result, these instabilities will be widely separated when $Re_{b}\gg 1$, as in the case of strongly stratified turbulence \cite{brethouwer2007}.

Despite the similar growth rates, we have discovered that the saturation level of these short-wave instabilities saturates at a level proportional to the aspect ratio $\delta^{3}$. For physical applications, $\delta \ll 1$ which suggests that the short-wave instabilities do not contribute significantly to the breakdown of the Lamb-Chaplygin dipole to turbulence.

Some unanswered questions still remain and provide future work. Structure-wise the short-wave instability is very different from that of the zigzag instability. As suggested by the sub-dominant mode analysis, there is likely a transition of leading eigenmodes between the zigzag and short-wave peaks. Furthermore, the nature of this transition is also interesting as it exhibits oscillations not observed in either the short-wave or zigzag instabilities. Briefly discussed were the waves behind the dipole which follow a simple scaling law. Investigations into waves behind vortices, especially those behind stratified vortices, has remained unexplored and would be of particular interest to numerical models. Such flow might resemble flow over terrain which predicts a similar scaling law for the angle of those waves. Finally, it would be interesting to investigate the short-wave instability in other systems, specifically adding in the effects of rotation. Such investigations of buoyancy length effects have illustrated the zigzag instability, which demonstrates the universality of the result. It is likely that such short-wave instabilities would be universal as well. 


% T I T L E   P A G E
% -------------------
% Last updated May 24, 2011, by Stephen Carr, IST-Client Services
% The title page is counted as page `i' but we need to suppress the
% page number.  We also don't want any headers or footers.
\pagestyle{empty}
\pagenumbering{roman}

% The contents of the title page are specified in the "titlepage"
% environment.
\begin{titlepage}
        \begin{center}
        \vspace*{1.0cm}

        \Huge
        {\bf Short-Wave Vortex Instabilities in Stratified Flow}

        \vspace*{1.0cm}

        \normalsize
        by \\

        \vspace*{1.0cm}

        \Large
        Luke Bovard \\

        \vspace*{3.0cm}

        \normalsize
        A thesis \\
        presented to the University of Waterloo \\ 
        in fulfillment of the \\
        thesis requirement for the degree of \\
        Master of Mathematics\\
        in \\
        Applied Mathematics \\

        \vspace*{2.0cm}

        Waterloo, Ontario, Canada, 2013 \\

        \vspace*{1.0cm}

        \copyright\ Luke Bovard 2013 \\
        \end{center}
\end{titlepage}

% The rest of the front pages should contain no headers and be numbered using Roman numerals starting with `ii'
\pagestyle{plain}
\setcounter{page}{2}

\cleardoublepage % Ends the current page and causes all figures and tables that have so far appeared in the input to be printed.
% In a two-sided printing style, it also makes the next page a right-hand (odd-numbered) page, producing a blank page if necessary.
 


% D E C L A R A T I O N   P A G E
% -------------------------------
  % The following is the sample Delaration Page as provided by the GSO
  % December 13th, 2006.  It is designed for an electronic thesis.
  \noindent
I hereby declare that I am the sole author of this thesis. This is a true copy of the thesis, including any required final revisions, as accepted by my examiners.

  \bigskip
  
  \noindent
I understand that my thesis may be made electronically available to the public.

\cleardoublepage
%\newpage

% A B S T R A C T
% ---------------

\begin{center}\textbf{Abstract}\end{center}
Stratified flow is the essential underlying physical model for atmospheric and stratified flow. In recent years, the study of stratified turbulence has been more thoroughly investigated due to its difference from classical turbulence. As a first step to investigating the mechanisms of turbulence, linear stability plays a critical role in determining under what conditions a flow remains stable or becomes turbulent. In the study of transition to stratified turbulence, a common vortex model, known as the Lamb-Chaplygin dipole, is used to investigate the conditions under which stratified flow transitions to turbulence. Numerous investigations have determined that a critical length scale, known as the buoyancy length, plays a key role in the breakdown and transition to stratified turbulence. At this buoyancy length scale, an instability unique to stratified flow, the zigzag instability emerges. However investigations into sub-buoyancy length scales have remained unexplored. In this thesis we discover and investigate a new instability of the Lamb-Chaplyin dipole that exists at the sub-buoyancy scale. Through numerical linear stability analysis we show that this short-wave instability exhibits growth rates similar to that of the zigzag instability. We conclude with nonlinear studies of this short-wave instability and demonstrate this new instability saturates at a level proportional to the aspect ratio. 
\cleardoublepage
%\newpage

% A C K N O W L E D G E M E N T S
% -------------------------------

\begin{center}\textbf{Acknowledgements}\end{center}
Thanks to my supervisor Dr. Michael Waite. Additional thanks goes to Dr. Marek Statsna and Dr. Francis Poulin. 

Financial support for this work was provided by the Natural Sciences and Engineering Research Council of Canada. Computations were made possible by the facilities of the Shared Hierarchical Academic Research Computing Network (SHARCNET) and Compute/Calcul Canada.

Obligatory thanks to everyone else who I have learnt from and talked to over all these year. There are too many to name, so I won't. Just know that if you knew me I'd probably thank you. This saves me having to remember people and just cover everyone in one full swoop.

The work for this thesis was written under the music of, in order of most played, Queen, Taylor Swift, The Beatles, Dire Straits, and Billy Joel. I kept statistics and it works out to be a few thousands plays of each, which is probably about 4 days straight of music per artists!

Richard you wanted me to add you punching Chris by mistake many years ago to this thesis. Here you go.

Special thanks to Patricia for making the last few months bearable. It made writing this thesis much easier. Thanks et merci y gracias por todo. 

Finally, and most importantly, I dedicate this thesis to the two best teachers I've ever had. Without both of them, who knows where I'd be now. 
\cleardoublepage
%\newpage

% D E D I C A T I O N
% -------------------

\begin{center}\textbf{Dedication}\end{center}
\begin{center}To Dr. Edward Vrscay and Dr. Jamal Sakhr.\end{center}
\cleardoublepage

%\newpage

% T A B L E   O F   C O N T E N T S
% ---------------------------------
\renewcommand\contentsname{Table of Contents}
\tableofcontents
\cleardoublepage
\phantomsection
%\newpage

% L I S T   O F   T A B L E S
% ---------------------------
%\addcontentsline{toc}{chapter}{List of Tables}
%\listoftables
%\cleardoublepage
%\phantomsection		% allows hyperref to link to the correct page
%%\newpage

% L I S T   O F   F I G U R E S
% -----------------------------
\addcontentsline{toc}{chapter}{List of Figures}
\listoffigures
\cleardoublepage
\phantomsection		% allows hyperref to link to the correct page
%\newpage

% L I S T   O F   S Y M B O L S
% -----------------------------
% To include a Nomenclature section
% \addcontentsline{toc}{chapter}{\textbf{Nomenclature}}
% \renewcommand{\nomname}{Nomenclature}
% \printglossary
% \cleardoublepage
% \phantomsection % allows hyperref to link to the correct page
% \newpage

% Change page numbering back to Arabic numerals
\pagenumbering{arabic}

